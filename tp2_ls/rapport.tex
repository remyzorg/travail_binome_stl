
\pdfminorversion 7
\pdfobjcompresslevel 3

\documentclass[a4paper]{article}
\special{papersize=210mm,297mm}
\usepackage[utf8]{inputenc}
\usepackage[T1]{fontenc}
\usepackage{cite}
\usepackage[francais]{babel}
\usepackage[bookmarks=false,colorlinks,linkcolor=blue]{hyperref}
\usepackage[top=3cm,bottom=2cm,left=3cm,right=2cm]{geometry}
\usepackage{graphicx}
\usepackage{subfig}
\usepackage{eso-pic}
\usepackage{array}
\usepackage{color}
\usepackage{url}
\usepackage{listings}
\usepackage{eurosym}
\usepackage{url}
\usepackage{textcomp}
\usepackage{fancyhdr} 
\usepackage{tikz}
\usetikzlibrary{automata,positioning}

\definecolor{lightgray}{gray}{0.9}

\title{Rapport de TP Lustre}
\author{Rémy \textsc{El-Sibaie Besognet} -- Roven \textsc{Gabriel}}

\newcommand{\HRule}{\rule{\linewidth}{0.5mm}}


\begin{document}

\maketitle

\section{Metronome}

\subsection{Code Lustre}
Afin d'implémenter le noeud lustre, nous supposons que le delai entre deux ticks
est éguale à $hz + 1 $. Ci-contre le programme résulant. % ou en annex
\begin{verbatim}
node metronome (reset : bool; delay : int) returns (tic, tac : bool);
  var hz,  n :  int; first,  state :  bool;  
let 
  
  hz = if reset then delay else (0 -> pre hz);
  first = (false -> pre first) or reset;
  n = if not first then 1 else 
        if reset or pre n = 0 then hz
        else pre n - 1;

  state = true -> if pre n = 0 then not pre state else pre state;

  tic = n = 0 and state;
  tac = n = 0 and not state;
  
  
  assert (delay >= 0);

tel 
\end{verbatim}

Nous simulons le fonctionnement du métronome dans 3 cas : 
\begin{itemize}
\item ticks d'horloge avant reset initial : \emph{1 à 4}
\item reset initial avec delai de 2 avec changement de delai à 0 : \emph{-1 à 14}
\item enchainement de reset avec delai de 0 puis delai de 1 : \emph{15 à 24}
\end{itemize}

\vspace{10mm}

% Sim2chro Locness  version 0.2 alpha (c) Mar 2004
% please report bugs, so I can report them as features ;-)
{
\def\lignefine{\linethickness{0.05pt}}
\def\ligneepaisse{\linethickness{2pt}}
\noindent
\setlength{\unitlength}{1mm}
\begin{picture}(145.6667,50.0000)(-19.0000,-5.0000)
\fboxsep 0pt
\lignefine
\color{black}
\multiput(0.0000,-5.0000)(4.3333,0){30}{\line(0,1){50.0000}}
\put(2.1667,44.0000){\scriptsize\makebox(0,0)[t]{1}}
\put(2.1667,-4.0000){\scriptsize\makebox(0,0)[b]{1}}
\put(6.5000,44.0000){\scriptsize\makebox(0,0)[t]{2}}
\put(6.5000,-4.0000){\scriptsize\makebox(0,0)[b]{2}}
\put(10.8333,44.0000){\scriptsize\makebox(0,0)[t]{3}}
\put(10.8333,-4.0000){\scriptsize\makebox(0,0)[b]{3}}
\put(15.1667,44.0000){\scriptsize\makebox(0,0)[t]{4}}
\put(15.1667,-4.0000){\scriptsize\makebox(0,0)[b]{4}}
\put(19.5000,44.0000){\scriptsize\makebox(0,0)[t]{-1}}
\put(19.5000,-4.0000){\scriptsize\makebox(0,0)[b]{-1}}
\put(23.8333,44.0000){\scriptsize\makebox(0,0)[t]{1}}
\put(23.8333,-4.0000){\scriptsize\makebox(0,0)[b]{1}}
\put(28.1667,44.0000){\scriptsize\makebox(0,0)[t]{2}}
\put(28.1667,-4.0000){\scriptsize\makebox(0,0)[b]{2}}
\put(32.5000,44.0000){\scriptsize\makebox(0,0)[t]{3}}
\put(32.5000,-4.0000){\scriptsize\makebox(0,0)[b]{3}}
\put(36.8333,44.0000){\scriptsize\makebox(0,0)[t]{4}}
\put(36.8333,-4.0000){\scriptsize\makebox(0,0)[b]{4}}
\put(41.1667,44.0000){\scriptsize\makebox(0,0)[t]{5}}
\put(41.1667,-4.0000){\scriptsize\makebox(0,0)[b]{5}}
\put(45.5000,44.0000){\scriptsize\makebox(0,0)[t]{6}}
\put(45.5000,-4.0000){\scriptsize\makebox(0,0)[b]{6}}
\put(49.8333,44.0000){\scriptsize\makebox(0,0)[t]{7}}
\put(49.8333,-4.0000){\scriptsize\makebox(0,0)[b]{7}}
\put(54.1667,44.0000){\scriptsize\makebox(0,0)[t]{8}}
\put(54.1667,-4.0000){\scriptsize\makebox(0,0)[b]{8}}
\put(58.5000,44.0000){\scriptsize\makebox(0,0)[t]{9}}
\put(58.5000,-4.0000){\scriptsize\makebox(0,0)[b]{9}}
\put(62.8333,44.0000){\scriptsize\makebox(0,0)[t]{10}}
\put(62.8333,-4.0000){\scriptsize\makebox(0,0)[b]{10}}
\put(67.1667,44.0000){\scriptsize\makebox(0,0)[t]{11}}
\put(67.1667,-4.0000){\scriptsize\makebox(0,0)[b]{11}}
\put(71.5000,44.0000){\scriptsize\makebox(0,0)[t]{12}}
\put(71.5000,-4.0000){\scriptsize\makebox(0,0)[b]{12}}
\put(75.8333,44.0000){\scriptsize\makebox(0,0)[t]{13}}
\put(75.8333,-4.0000){\scriptsize\makebox(0,0)[b]{13}}
\put(80.1667,44.0000){\scriptsize\makebox(0,0)[t]{14}}
\put(80.1667,-4.0000){\scriptsize\makebox(0,0)[b]{14}}
\put(84.5000,44.0000){\scriptsize\makebox(0,0)[t]{15}}
\put(84.5000,-4.0000){\scriptsize\makebox(0,0)[b]{15}}
\put(88.8333,44.0000){\scriptsize\makebox(0,0)[t]{16}}
\put(88.8333,-4.0000){\scriptsize\makebox(0,0)[b]{16}}
\put(93.1667,44.0000){\scriptsize\makebox(0,0)[t]{17}}
\put(93.1667,-4.0000){\scriptsize\makebox(0,0)[b]{17}}
\put(97.5000,44.0000){\scriptsize\makebox(0,0)[t]{18}}
\put(97.5000,-4.0000){\scriptsize\makebox(0,0)[b]{18}}
\put(101.8333,44.0000){\scriptsize\makebox(0,0)[t]{19}}
\put(101.8333,-4.0000){\scriptsize\makebox(0,0)[b]{19}}
\put(106.1667,44.0000){\scriptsize\makebox(0,0)[t]{20}}
\put(106.1667,-4.0000){\scriptsize\makebox(0,0)[b]{20}}
\put(110.5000,44.0000){\scriptsize\makebox(0,0)[t]{21}}
\put(110.5000,-4.0000){\scriptsize\makebox(0,0)[b]{21}}
\put(114.8333,44.0000){\scriptsize\makebox(0,0)[t]{22}}
\put(114.8333,-4.0000){\scriptsize\makebox(0,0)[b]{22}}
\put(119.1667,44.0000){\scriptsize\makebox(0,0)[t]{23}}
\put(119.1667,-4.0000){\scriptsize\makebox(0,0)[b]{23}}
\put(123.5000,44.0000){\scriptsize\makebox(0,0)[t]{24}}
\put(123.5000,-4.0000){\scriptsize\makebox(0,0)[b]{24}}
\put(-1.0000,32.0000){\line(1,0){127.6667}}
\put(-1.0000,38.0000){\line(1,0){127.6667}}
\put(-1.0000,18.0000){\line(1,0){127.6667}}
\put(-1.0000,30.0000){\line(1,0){127.6667}}
\put(-1.0000,10.0000){\line(1,0){127.6667}}
\put(-1.0000,16.0000){\line(1,0){127.6667}}
\put(-1.0000,2.0000){\line(1,0){127.6667}}
\put(-1.0000,8.0000){\line(1,0){127.6667}}
\ligneepaisse
\color{blue}
\put(-1.0000,35.0000){\color{blue}\normalsize\makebox(0,0)[r]{reset}}
\put(26.0000,38.0000){\line(0,-1){6.0000}}
\put(60.6667,32.0000){\line(0,1){6.0000}}
\put(65.0000,38.0000){\line(0,-1){6.0000}}
\put(82.3333,32.0000){\line(0,1){6.0000}}
\put(0.0000,32.0000){\line(1,0){4.3333}}
\put(4.3333,32.0000){\line(1,0){4.3333}}
\put(8.6667,32.0000){\line(1,0){4.3333}}
\put(13.0000,32.0000){\line(1,0){4.3333}}
\put(21.6667,38.0000){\line(1,0){4.3333}}
\put(26.0000,32.0000){\line(1,0){4.3333}}
\put(30.3333,32.0000){\line(1,0){4.3333}}
\put(34.6667,32.0000){\line(1,0){4.3333}}
\put(39.0000,32.0000){\line(1,0){4.3333}}
\put(43.3333,32.0000){\line(1,0){4.3333}}
\put(47.6667,32.0000){\line(1,0){4.3333}}
\put(52.0000,32.0000){\line(1,0){4.3333}}
\put(56.3333,32.0000){\line(1,0){4.3333}}
\put(60.6667,38.0000){\line(1,0){4.3333}}
\put(65.0000,32.0000){\line(1,0){4.3333}}
\put(69.3333,32.0000){\line(1,0){4.3333}}
\put(73.6667,32.0000){\line(1,0){4.3333}}
\put(78.0000,32.0000){\line(1,0){4.3333}}
\put(82.3333,38.0000){\line(1,0){4.3333}}
\put(86.6667,38.0000){\line(1,0){4.3333}}
\put(91.0000,38.0000){\line(1,0){4.3333}}
\put(95.3333,38.0000){\line(1,0){4.3333}}
\put(99.6667,38.0000){\line(1,0){4.3333}}
\put(104.0000,38.0000){\line(1,0){4.3333}}
\put(108.3333,38.0000){\line(1,0){4.3333}}
\put(112.6667,38.0000){\line(1,0){4.3333}}
\put(117.0000,38.0000){\line(1,0){4.3333}}
\put(121.3333,38.0000){\line(1,0){4.3333}}
\color{blue}
\put(-1.0000,24.0000){\color{blue}\normalsize\makebox(0,0)[r]{delay}}
\put(60.6667,30.0000){\line(0,-1){12.0000}}
\put(104.0000,18.0000){\line(0,1){6.0000}}
\put(0.0000,18.0000){\line(1,0){4.3333}}
\put(2.1667,19.0000){\color{blue}\scriptsize\makebox(0,0)[b]{0}}
\put(4.3333,18.0000){\line(1,0){4.3333}}
\put(6.5000,19.0000){\color{blue}\scriptsize\makebox(0,0)[b]{0}}
\put(8.6667,18.0000){\line(1,0){4.3333}}
\put(10.8333,19.0000){\color{blue}\scriptsize\makebox(0,0)[b]{0}}
\put(13.0000,18.0000){\line(1,0){4.3333}}
\put(15.1667,19.0000){\color{blue}\scriptsize\makebox(0,0)[b]{0}}
\put(21.6667,30.0000){\line(1,0){4.3333}}
\put(23.8333,29.0000){\color{blue}\scriptsize\makebox(0,0)[t]{2}}
\put(26.0000,30.0000){\line(1,0){4.3333}}
\put(28.1667,29.0000){\color{blue}\scriptsize\makebox(0,0)[t]{2}}
\put(30.3333,30.0000){\line(1,0){4.3333}}
\put(32.5000,29.0000){\color{blue}\scriptsize\makebox(0,0)[t]{2}}
\put(34.6667,30.0000){\line(1,0){4.3333}}
\put(36.8333,29.0000){\color{blue}\scriptsize\makebox(0,0)[t]{2}}
\put(39.0000,30.0000){\line(1,0){4.3333}}
\put(41.1667,29.0000){\color{blue}\scriptsize\makebox(0,0)[t]{2}}
\put(43.3333,30.0000){\line(1,0){4.3333}}
\put(45.5000,29.0000){\color{blue}\scriptsize\makebox(0,0)[t]{2}}
\put(47.6667,30.0000){\line(1,0){4.3333}}
\put(49.8333,29.0000){\color{blue}\scriptsize\makebox(0,0)[t]{2}}
\put(52.0000,30.0000){\line(1,0){4.3333}}
\put(54.1667,29.0000){\color{blue}\scriptsize\makebox(0,0)[t]{2}}
\put(56.3333,30.0000){\line(1,0){4.3333}}
\put(58.5000,29.0000){\color{blue}\scriptsize\makebox(0,0)[t]{2}}
\put(60.6667,18.0000){\line(1,0){4.3333}}
\put(62.8333,19.0000){\color{blue}\scriptsize\makebox(0,0)[b]{0}}
\put(65.0000,18.0000){\line(1,0){4.3333}}
\put(67.1667,19.0000){\color{blue}\scriptsize\makebox(0,0)[b]{0}}
\put(69.3333,18.0000){\line(1,0){4.3333}}
\put(71.5000,19.0000){\color{blue}\scriptsize\makebox(0,0)[b]{0}}
\put(73.6667,18.0000){\line(1,0){4.3333}}
\put(75.8333,19.0000){\color{blue}\scriptsize\makebox(0,0)[b]{0}}
\put(78.0000,18.0000){\line(1,0){4.3333}}
\put(80.1667,19.0000){\color{blue}\scriptsize\makebox(0,0)[b]{0}}
\put(82.3333,18.0000){\line(1,0){4.3333}}
\put(84.5000,19.0000){\color{blue}\scriptsize\makebox(0,0)[b]{0}}
\put(86.6667,18.0000){\line(1,0){4.3333}}
\put(88.8333,19.0000){\color{blue}\scriptsize\makebox(0,0)[b]{0}}
\put(91.0000,18.0000){\line(1,0){4.3333}}
\put(93.1667,19.0000){\color{blue}\scriptsize\makebox(0,0)[b]{0}}
\put(95.3333,18.0000){\line(1,0){4.3333}}
\put(97.5000,19.0000){\color{blue}\scriptsize\makebox(0,0)[b]{0}}
\put(99.6667,18.0000){\line(1,0){4.3333}}
\put(101.8333,19.0000){\color{blue}\scriptsize\makebox(0,0)[b]{0}}
\put(104.0000,24.0000){\line(1,0){4.3333}}
\put(106.1667,23.0000){\color{blue}\scriptsize\makebox(0,0)[t]{1}}
\put(108.3333,24.0000){\line(1,0){4.3333}}
\put(110.5000,23.0000){\color{blue}\scriptsize\makebox(0,0)[t]{1}}
\put(112.6667,24.0000){\line(1,0){4.3333}}
\put(114.8333,23.0000){\color{blue}\scriptsize\makebox(0,0)[t]{1}}
\put(117.0000,24.0000){\line(1,0){4.3333}}
\put(119.1667,23.0000){\color{blue}\scriptsize\makebox(0,0)[t]{1}}
\put(121.3333,24.0000){\line(1,0){4.3333}}
\put(123.5000,23.0000){\color{blue}\scriptsize\makebox(0,0)[t]{1}}
\color{red}
\put(-1.0000,13.0000){\color{red}\normalsize\makebox(0,0)[r]{tic}}
\put(30.3333,10.0000){\line(0,1){6.0000}}
\put(34.6667,16.0000){\line(0,-1){6.0000}}
\put(56.3333,10.0000){\line(0,1){6.0000}}
\put(60.6667,16.0000){\line(0,-1){6.0000}}
\put(65.0000,10.0000){\line(0,1){6.0000}}
\put(69.3333,16.0000){\line(0,-1){6.0000}}
\put(73.6667,10.0000){\line(0,1){6.0000}}
\put(78.0000,16.0000){\line(0,-1){6.0000}}
\put(82.3333,10.0000){\line(0,1){6.0000}}
\put(86.6667,16.0000){\line(0,-1){6.0000}}
\put(91.0000,10.0000){\line(0,1){6.0000}}
\put(95.3333,16.0000){\line(0,-1){6.0000}}
\put(99.6667,10.0000){\line(0,1){6.0000}}
\put(104.0000,16.0000){\line(0,-1){6.0000}}
\put(0.0000,10.0000){\line(1,0){4.3333}}
\put(4.3333,10.0000){\line(1,0){4.3333}}
\put(8.6667,10.0000){\line(1,0){4.3333}}
\put(13.0000,10.0000){\line(1,0){4.3333}}
\put(21.6667,10.0000){\line(1,0){4.3333}}
\put(26.0000,10.0000){\line(1,0){4.3333}}
\put(30.3333,16.0000){\line(1,0){4.3333}}
\put(34.6667,10.0000){\line(1,0){4.3333}}
\put(39.0000,10.0000){\line(1,0){4.3333}}
\put(43.3333,10.0000){\line(1,0){4.3333}}
\put(47.6667,10.0000){\line(1,0){4.3333}}
\put(52.0000,10.0000){\line(1,0){4.3333}}
\put(56.3333,16.0000){\line(1,0){4.3333}}
\put(60.6667,10.0000){\line(1,0){4.3333}}
\put(65.0000,16.0000){\line(1,0){4.3333}}
\put(69.3333,10.0000){\line(1,0){4.3333}}
\put(73.6667,16.0000){\line(1,0){4.3333}}
\put(78.0000,10.0000){\line(1,0){4.3333}}
\put(82.3333,16.0000){\line(1,0){4.3333}}
\put(86.6667,10.0000){\line(1,0){4.3333}}
\put(91.0000,16.0000){\line(1,0){4.3333}}
\put(95.3333,10.0000){\line(1,0){4.3333}}
\put(99.6667,16.0000){\line(1,0){4.3333}}
\put(104.0000,10.0000){\line(1,0){4.3333}}
\put(108.3333,10.0000){\line(1,0){4.3333}}
\put(112.6667,10.0000){\line(1,0){4.3333}}
\put(117.0000,10.0000){\line(1,0){4.3333}}
\put(121.3333,10.0000){\line(1,0){4.3333}}
\color{red}
\put(-1.0000,5.0000){\color{red}\normalsize\makebox(0,0)[r]{tac}}
\put(43.3333,2.0000){\line(0,1){6.0000}}
\put(47.6667,8.0000){\line(0,-1){6.0000}}
\put(60.6667,2.0000){\line(0,1){6.0000}}
\put(65.0000,8.0000){\line(0,-1){6.0000}}
\put(69.3333,2.0000){\line(0,1){6.0000}}
\put(73.6667,8.0000){\line(0,-1){6.0000}}
\put(78.0000,2.0000){\line(0,1){6.0000}}
\put(82.3333,8.0000){\line(0,-1){6.0000}}
\put(86.6667,2.0000){\line(0,1){6.0000}}
\put(91.0000,8.0000){\line(0,-1){6.0000}}
\put(95.3333,2.0000){\line(0,1){6.0000}}
\put(99.6667,8.0000){\line(0,-1){6.0000}}
\put(0.0000,2.0000){\line(1,0){4.3333}}
\put(4.3333,2.0000){\line(1,0){4.3333}}
\put(8.6667,2.0000){\line(1,0){4.3333}}
\put(13.0000,2.0000){\line(1,0){4.3333}}
\put(21.6667,2.0000){\line(1,0){4.3333}}
\put(26.0000,2.0000){\line(1,0){4.3333}}
\put(30.3333,2.0000){\line(1,0){4.3333}}
\put(34.6667,2.0000){\line(1,0){4.3333}}
\put(39.0000,2.0000){\line(1,0){4.3333}}
\put(43.3333,8.0000){\line(1,0){4.3333}}
\put(47.6667,2.0000){\line(1,0){4.3333}}
\put(52.0000,2.0000){\line(1,0){4.3333}}
\put(56.3333,2.0000){\line(1,0){4.3333}}
\put(60.6667,8.0000){\line(1,0){4.3333}}
\put(65.0000,2.0000){\line(1,0){4.3333}}
\put(69.3333,8.0000){\line(1,0){4.3333}}
\put(73.6667,2.0000){\line(1,0){4.3333}}
\put(78.0000,8.0000){\line(1,0){4.3333}}
\put(82.3333,2.0000){\line(1,0){4.3333}}
\put(86.6667,8.0000){\line(1,0){4.3333}}
\put(91.0000,2.0000){\line(1,0){4.3333}}
\put(95.3333,8.0000){\line(1,0){4.3333}}
\put(99.6667,2.0000){\line(1,0){4.3333}}
\put(104.0000,2.0000){\line(1,0){4.3333}}
\put(108.3333,2.0000){\line(1,0){4.3333}}
\put(112.6667,2.0000){\line(1,0){4.3333}}
\put(117.0000,2.0000){\line(1,0){4.3333}}
\put(121.3333,2.0000){\line(1,0){4.3333}}
\end{picture}
}


\subsection{Automate}
L'automate déduit du code lustre ayant été trop optimisé, nous avons observé 
l'automate du programme C généré sans optimisation. Celui-ci se résume à un état ne 
contenant qu'une transission.\\

 \begin{tikzpicture}[%
    >=stealth,
    node distance=3cm,
    on grid,
    auto,
    new state/.style={draw=green!50,very thick,fill=green!20},
    new edge/.style={draw=green,very thick}
  ]
    \node[state,initial] (I)      {0};
    \path[->] 
        (I) edge [loop right] node[align=left] { 
        $\tild$reset,$\tild$delay \textbf{/} \\
        first = (init? false : first) || reset \\
        hz = reset? delay : (init? 0 : hz) \\
        state = init? true : ((n == 0)? (not state) : state) \\
        init = false \\
        n = (not first)? 1 : ((reset || (n == 0))? hz : (n - 1)) \\
        \\
        tic = (n == 0) $\wedge$ state \\
        tac = (n == 0) $\wedge$ (not state)
        } (I);
  \end{tikzpicture}


%\begin{verbatim}
%      first = (init? _false : first) || reset;
%      hz = reset? delay : (init? 0 : hz);
%      state = init? _true : ((n == 0)? (!state) : state);
%      init = _false;
%      n = (!first)? 1 : ((reset || (n == 0))? hz : (
%n - 1));
%
%      tic = (n == 0) && state;
%      tac = (n == 0) && (!state);
%\end{verbatim}


Après compilation avec optimisation, l'automate généré contient 5 états.

 \begin{tikzpicture}[%
    >=stealth,
    node distance=4cm,
    on grid,
    auto,
    new state/.style={draw=green!50,very thick,fill=green!20},
    new edge/.style={draw=green,very thick}
  ]
    \node[state] (S2)        {S2};
    \node[state,initial] (S0) [left of=S2] {S0};
    \node[state] (S1) [above of=S2] {S1};
    \node[state] (S4) [right of=S2] {S4};
    \node[state] (S3) [below of=S2] {S3};
    \path[->] 
        (S0) edge [bend left] node[align=left] { 
            reset, n=0 / tic, $\neg$tac \\
            reset / $\neg$tic, $\neg$tac } (S1)
        (S0) edge [bend right] node[below,align=left] { 
            $\neg$reset / $\neg$tic, $\neg$tac } (S2)

        (S1) edge [bend left=90,looseness=3.1] node[right, align=left] { 
            n=0 / \\ $\neg$tic, tac } (S3)
        (S1) edge [loop above] node[align=left] { 
            / $\neg$tic, $\neg$ tac \\
            n=0, state / tic, $\neg$tac } (S1)
        

        (S2) edge [bend right] node[below left, align=left] {
            reset, n=0 / \\ $\neg$tic, tac } (S3)
        (S2) edge [bend left] node[left,align=left] { 
            reset, n=0 / \\ tic, $\neg$tac } (S1)
        (S2) edge [bend left] node[above,align=left] {
            n=0 / $\neg$tic, tac } (S4)
        (S2) edge [loop left] node[align=left] { 
            / $\neg$tic,$\neg$tac } (S2)

        (S3) edge [bend right=85,looseness=2.3] node[right,align=left] {
            n=0 / \\ tic, $\neg$tac } (S1)
        (S3) edge [loop below] node[align=left] {
            / $\neg$tic,$\neg$tac} (S3)

        (S4) edge [bend left] node[left, align=left] {
            reset, n=0 / \\ $\neg$tic, tac } (S3)
        (S4) edge [bend right] node[below left,align=left] { 
            reset, n=0 / \\ tic, $\neg$tac } (S1)
        (S4) edge [bend left] node[below,align=left] {
            n=0 / $\neg$tic, tac } (S2)
        (S4) edge [loop left] node[align=left] { 
            / $\neg$tic,$\neg$tac } (S4);
  \end{tikzpicture}

%\begin{verbatim}
%   case 0:
%      ctx->_V5 = (ctx->_V4 == 0);
%      if(ctx->_V1){
%         ctx->_V6 = ctx->_V2;
%         ctx->_V4 = ctx->_V6;
%      } else {
%         ctx->_V6 = 0;
%         ctx->_V4 = 1; // n
%      }
%      ctx->_V3 = (ctx->_V4 == 0);
%      if(ctx->_V3){ // n = 0
%         ctx->_V7 = _true; // tic
%         metronome_O_tic(ctx->client_data, ctx->_V7);
%      } else {
%         ctx->_V7 = _false; // tic
%         metronome_O_tic(ctx->client_data, ctx->_V7);
%      }
%      ctx->_V8 = _false; // tac
%      metronome_O_tac(ctx->client_data, ctx->_V8);
%      if(ctx->_V1){
%         ctx->current_state = 1; break;
%      } else {
%         ctx->current_state = 2; break;
%      }
%   break;
%   
%   case 1:
%      ctx->_V5 = (ctx->_V4 == 0);
%      if(ctx->_V1){
%         ctx->_V6 = ctx->_V2;
%         ctx->_V4 = ctx->_V6;
%      } else {
%         if(ctx->_V5){
%            ctx->_V4 = ctx->_V6;
%         } else {
%            ctx->_V4 = (ctx->_V4 - 1);
%         }
%      }
%      ctx->_V3 = (ctx->_V4 == 0);
%      if(ctx->_V3){
%         if(ctx->_V5){
%            ctx->_V7 = _false;
%            metronome_O_tic(ctx->client_data, ctx->_V7);
%            ctx->_V8 = _true;
%            metronome_O_tac(ctx->client_data, ctx->_V8);
%         } else {
%            ctx->_V7 = _true;
%            metronome_O_tic(ctx->client_data, ctx->_V7);
%            ctx->_V8 = _false;
%            metronome_O_tac(ctx->client_data, ctx->_V8);
%         }
%      } else {
%         ctx->_V7 = _false;
%         metronome_O_tic(ctx->client_data, ctx->_V7);
%         ctx->_V8 = _false;
%         metronome_O_tac(ctx->client_data, ctx->_V8);
%      }
%      if(ctx->_V5){
%         ctx->current_state = 3; break;
%      } else {
%         ctx->current_state = 1; break;
%      }
%   break;
%   
%   case 2:
%      ctx->_V5 = (ctx->_V4 == 0);
%      if(ctx->_V1){
%         ctx->_V6 = ctx->_V2;
%         ctx->_V4 = ctx->_V6;
%      } else {
%         ctx->_V4 = 1;
%      }
%      ctx->_V3 = (ctx->_V4 == 0);
%      if(ctx->_V3){
%         if(ctx->_V5){
%            ctx->_V7 = _false;
%            metronome_O_tic(ctx->client_data, ctx->_V7);
%            ctx->_V8 = _true;
%            metronome_O_tac(ctx->client_data, ctx->_V8);
%         } else {
%            ctx->_V7 = _true;
%            metronome_O_tic(ctx->client_data, ctx->_V7);
%            ctx->_V8 = _false;
%            metronome_O_tac(ctx->client_data, ctx->_V8);
%          else {
%         ctx->_V7 = _false;
%         metronome_O_tic(ctx->client_data, ctx->_V7);
%         ctx->_V8 = _false;
%         metronome_O_tac(ctx->client_data, ctx->_V8);
%      }
%      if(ctx->_V1){
%         if(ctx->_V5){
%            ctx->current_state = 3; break;
%         } else {
%            ctx->current_state = 1; break;
%         }
%      } else {
%         if(ctx->_V5){
%            ctx->current_state = 4; break;
%         } else {
%            ctx->current_state = 2; break;
%         }
%      }
%   break;
%   
%   case 3:
%      ctx->_V5 = (ctx->_V4 == 0);
%      if(ctx->_V1){
%         ctx->_V6 = ctx->_V2;
%         ctx->_V4 = ctx->_V6;
%      } else {
%         if(ctx->_V5){
%            ctx->_V4 = ctx->_V6;
%         } else {
%            ctx->_V4 = (ctx->_V4 - 1);
%         }
%      }
%      ctx->_V3 = (ctx->_V4 == 0);
%      if(ctx->_V3){
%         if(ctx->_V5){
%            ctx->_V7 = _true;
%            metronome_O_tic(ctx->client_data, ctx->_V7);
%            ctx->_V8 = _false;
%            metronome_O_tac(ctx->client_data, ctx->_V8);
%         } else {
%            ctx->_V7 = _false;
%            metronome_O_tic(ctx->client_data, ctx->_V7);
%            ctx->_V8 = _true;
%            metronome_O_tac(ctx->client_data, ctx->_V8);
%         }
%      } else {
%         ctx->_V7 = _false;
%         metronome_O_tic(ctx->client_data, ctx->_V7);
%         ctx->_V8 = _false;
%         metronome_O_tac(ctx->client_data, ctx->_V8);
%      }
%      if(ctx->_V5){
%         ctx->current_state = 1; break;
%      } else {
%         ctx->current_state = 3; break;
%      }
%   break;
%   
%   case 4:
%      ctx->_V5 = (ctx->_V4 == 0);
%      if(ctx->_V1){
%         ctx->_V6 = ctx->_V2;
%         ctx->_V4 = ctx->_V6;
%      } else {
%         ctx->_V4 = 1;
%      }
%      ctx->_V3 = (ctx->_V4 == 0);
%      if(ctx->_V3){
%         if(ctx->_V5){
%            ctx->_V7 = _true;
%            metronome_O_tic(ctx->client_data, ctx->_V7);
%            ctx->_V8 = _false;
%            metronome_O_tac(ctx->client_data, ctx->_V8);
%         } else {
%            ctx->_V7 = _false;
%            metronome_O_tic(ctx->client_data, ctx->_V7);
%            ctx->_V8 = _true;
%            metronome_O_tac(ctx->client_data, ctx->_V8);
%         }
%      } else {
%         ctx->_V7 = _false;
%         metronome_O_tic(ctx->client_data, ctx->_V7);
%         ctx->_V8 = _false;
%         metronome_O_tac(ctx->client_data, ctx->_V8);
%      }
%      if(ctx->_V1){
%         if(ctx->_V5){
%            ctx->current_state = 1; break;
%         } else {
%            ctx->current_state = 3; break;
%         }
%      } else {
%         if(ctx->_V5){
%            ctx->current_state = 2; break;
%         } else {
%            ctx->current_state = 4; break;
%         }
%      }
%   break;
%\end{verbatim}

\subsection{Analyse de l'automate}

L'automate implicite qui ne contenait qu'une seule transition comportant toute la
logique du programme, l'automate optimiser contient 4 états de plus et des
transitions plus ciblés. En effet l'automate implicite a été déplié selon les
variables internes \texttt{first} et \texttt{state}. On remarque très bien
l'alternance des \texttt{tic} et \texttt{tac} entre deux états. 

Cependant on notera que cette automate indique des transitions impossibles comme 
le passage de \texttt{S2} à \texttt{S4} et réciproquement avec l'envoie de 
\texttt{tic} ou \texttt{tac} sans réception du premier \texttt{reset}. Cette 
anomalie est lié au fait que dans notre programme nous forçons le \texttt{n} à 1
tant qu'il y a first ce qui implique qu'il n'y aura jamais $\texttt{n} = 0$. Cette
subtilité n'est pas capturé par le programme de génération de code lustre.\\

\subsection{Vérification avec observateur}
Pour chaque propriétée, nous construisons l'automate et le noeud lustre 
observateur. \\

\textbf{P1} : A tout instant, on ne peut avoir simultanément tic et tac :\\

 \begin{tikzpicture}[%
    >=stealth,
    node distance=4cm,
    on grid,
    auto,
    new state/.style={draw=green!50,very thick,fill=green!20},
    new edge/.style={draw=green,very thick}
  ]
    \node[state,initial] (S0)    {S0};
    \path[->] 
        (S0) edge [loop above] node[align=left] { 
            / ok } (S0)
        (S0) edge [loop below] node[align=left] { 
            tic, tac / $\neg$ok } (S0);
  \end{tikzpicture}

\begin{verbatim}
node observerP1 (tic,  tac : bool) returns (res :  bool);
let
  res = not (tic and tac);
tel
\end{verbatim}

\textbf{P2} : On ne peut jamais avoir deux tic consécutifs (il y a exactement un tac entre deux tics).

\emph{\textbf{Remarque} : dans notre implémentation le métronome commence toujours
 par \texttt{tic}}.\\

 \begin{tikzpicture}[%
    >=stealth,
    node distance=4cm,
    on grid,
    auto,
    new state/.style={draw=green!50,very thick,fill=green!20},
    new edge/.style={draw=green,very thick}
  ]
    \node[state,initial] (S0)    {S0};
    \node[state] (S1) [right of=S0]   {S1};
    \path[->] 
        (S0) edge [loop above] node[align=left] { 
            tac / $\neg$ok } (S0)
        (S0) edge [bend left] node[align=left] { 
            tic / ok } (S1)
        (S1) edge [loop above] node[align=left] { 
            tic / $\neg$ok } (S1)
        (S1) edge [bend left] node[align=left] { 
            tac / ok } (S0);
  \end{tikzpicture}

\begin{verbatim}
node observerP2 (tic, tac :  bool) returns (res :  bool);
var last_state : bool;
let
  last_state = 
    if tic then true else if tac then false else false -> pre last_state;
  res =  
    not (tic and pre last_state) or 
    not (tac and not pre last_state);
tel
\end{verbatim}

\textbf{P3} : Les occurrences consécutives de deux tic (s'il n'y a pas de reset entre eux) sont séparées par exactement 2 * delay ticks de l'horloge de base.

\emph{\textbf{Remarque} : dans notre implémentation le délai entre un tic et tac est
égale à $\texttt{delay} + 1$. Donc la propriété à vérifié est corrigé à $2 * 
(\texttt{delay} + 1)$}.\\

 \begin{tikzpicture}[%
    >=stealth,
    node distance=4cm,
    on grid,
    auto,
    new state/.style={draw=green!50,very thick,fill=green!20},
    new edge/.style={draw=green,very thick}
  ]
    \node[state,initial] (S0)    {S0};
    \node[state] (S1) [right of=S0]   {S1};
    \path[->] 
        (S0) edge [loop above] node[align=left] { 
            $\neg$tic / ok } (S0)
        (S0) edge [bend left] node[align=left] { 
            tic / ok } (S1)
        (S1) edge [loop above] node[align=left] { 
            $\neg$tic / ok } (S1)
        (S1) edge [bend right=10] node[align=left] { 
            reset / ok } (S0)
        (S1) edge [bend left] node[align=left] { 
            tic, n $= 2 * (delay + 1)$ / ok \\
            tic, n $\neq 2 * (delay + 1)$ / $\neg$ok } (S0);
  \end{tikzpicture}


\begin{verbatim}
node observerP3 (tic,  reset :  bool; hz :  int) returns (res :  bool);
var counter :  int; waiting_second :  bool;
let
  assert(hz >=  0);

  waiting_second = false ->
    if reset then false else
      if tic then not pre waiting_second else
        pre waiting_second;
  counter = if waiting_second then 0 -> pre counter + 1 else 0;

  res = true -> if pre waiting_second and tic then pre counter = 2 * (hz + 1) 
        else true;
tel
\end{verbatim}

L'ensemble des observateurs est synchronisé par le noeud suivant : 

\begin{verbatim}
node organizer (reset :  bool; delay :  int) returns (res1, res2, res3 :  bool);
var hz : int; tic, tac : bool;
let 
  
  hz = if reset then delay else (0 -> pre hz);
  tic, tac = metronome(reset,  delay);

  res1 = observerP1(tic, tac);
  res2 = observerP2(tic, tac);
  res3 = observerP3(tic, reset, hz);
tel
\end{verbatim}

\subsubsection{Vérification lesar}
L'observateur doit vérifier la propriété \texttt{res1 and res2 and res3}. Nous devons
ajouter l'hypothèse que \texttt{delay} est strictement supérieur à zéro. Cette 
contrainte est exprimée sous forme d'une assertion.

Les propriétés P1 et P2 sont jugées VRAIE par lesar. Cependant la propriété P3 est
jugée fausse avec la trace suivante : 
\begin{verbatim}
--Pollux Version 2.4

DIAGNOSIS:
LOCAL VARIABLE C0 IS
(delay)>=(0)
LOCAL VARIABLE C1 IS
(V50_V32_hz)>=(0)
LOCAL VARIABLE C2 IS
(last(V58_V155_counter))=((2)*((V50_V32_hz)+(1)))
LOCAL VARIABLE C3 IS
(V54_V124_n)=(0)
LOCAL VARIABLE C4 IS
(last(V54_V124_n))=(0)

--- TRANSITION 1 ---
C0 and C1
--- TRANSITION 2 ---
C0 and C1 and not reset and C3 and not C4
--- TRANSITION 3 ---
not C2 and C3 and not C4
FALSE PROPERTY
\end{verbatim}

Lexar indique un problème lorsque les variables internes C2 et C4 sont à faux, 
c'est à dire \texttt{pre counter = 2 * (hz + 1)} et \texttt{n = 0} (tic). 

Nous avons testé l'observateur sous différentes conditions (avant reset, reset 
pendant tic, reset après etc), la variable interne \texttt{waiting\_second}
alterne correctement, le resultat calculé $count = 2 * (hz + 1)$ est toujours vrai.

Toutes nos tentatives de reproduire le problème ont été infructueuses, nous supposons
que le problème peut être lié à notre gestion de \texttt{first} et de l'implication
implicite $first \Rightarrow n = 1$. \\

En conclusion la propriété générale n'est donc pas vérifié formellement par notre noeud lustre.

\section{Arbitre de McMillan}

\subsection{Modelisation lustre}
A partir des schemas donnés, nous implémentons un noeud \texttt{cell} représentant 
une seule cellule puis nous implémentons le noeud \texttt{arbitre3} qui représente
un arbitre à trois céllules.\\

\begin{verbatim}
node cell (tk_in, ovr_in, gr_in, rq_in :  bool) returns
  (ack_out, ovr_out, tk_out, gr_out : bool);
var w, w_and_t : bool;  
let 
  tk_out = false -> pre tk_in;  

  w = false -> rq_in and (tk_out or pre w);

  w_and_t = pre w and tk_out;

  ovr_out = w_and_t or ovr_in;

  gr_out = not rq_in and gr_in;
  
  ack_out = rq_in and (w_and_t or gr_in); 

tel

node arbitre_3(rq_in_1, rq_in_2, rq_in_3 : bool) returns 
  (ack_out_1, ack_out_2, ack_out_3, gr_out : bool);
var tk_in_1, ovr_in_1,
  tk_out_1, ovr_out_1, gr_in_1,
    
  tk_out_2, ovr_out_2, gr_out_2,

  tk_out_3, ovr_out_3, gr_out_3 : bool;
let
  tk_in_1 = true -> tk_out_3;
  ovr_in_1 = false;

  ack_out_1, ovr_out_1, tk_out_1, gr_out = 
    cell(tk_in_1, ovr_in_1, gr_out_2,  rq_in_1);

  ack_out_2, ovr_out_2, tk_out_2, gr_out_2 = 
    cell(tk_out_1, ovr_out_1, gr_out_3,  rq_in_2);

  ack_out_3, ovr_out_3, tk_out_3, gr_out_3 = 
    cell(tk_out_2, ovr_out_2, not ovr_out_3,  rq_in_3);
tel
\end{verbatim}

\subsubsection{Simulations}
Nous simulons l'arbitre de trois cellule sur les cas de figure suivants : 
%  à un seule requête continue (sur plusieurs cycles), à deux requêtes toujours exclusives, à deux requêtes non exclusives, à trois requêtes sporadiques et à trois requêtes continues. Vous étudierez également les quelques cycles suivant l'initialisation.
\begin{itemize}
\item simple étude des évenements après initialisation : \emph{1 à 5}
\item soumission d'une requête en continue : \emph{6 à 15}
\item soumission de deux requêtes exclusives : \emph{16 à 30}
\item soumission de deux requêtes non exclusives : \emph{31 à 41} et \emph{44 à 60}
\item soumission de trois requêtes : \emph{62 à 73}
\item soumission de trois requêtes en continue : \emph{76 à 90}
\end{itemize}

\vspace{10mm}

% Sim2chro Locness  version 0.2 alpha (c) Mar 2004
% please report bugs, so I can report them as features ;-)
{
\def\lignefine{\linethickness{0.05pt}}
\def\ligneepaisse{\linethickness{2pt}}
\noindent
\setlength{\unitlength}{1mm}
\begin{picture}(150.0000,50.0000)(-19.0000,-5.0000)
\fboxsep 0pt
\lignefine
\color{black}
\multiput(0.0000,-5.0000)(4.3333,0){31}{\line(0,1){50.0000}}
\put(2.1667,44.0000){\scriptsize\makebox(0,0)[t]{1}}
\put(2.1667,-4.0000){\scriptsize\makebox(0,0)[b]{1}}
\put(6.5000,44.0000){\scriptsize\makebox(0,0)[t]{2}}
\put(6.5000,-4.0000){\scriptsize\makebox(0,0)[b]{2}}
\put(10.8333,44.0000){\scriptsize\makebox(0,0)[t]{3}}
\put(10.8333,-4.0000){\scriptsize\makebox(0,0)[b]{3}}
\put(15.1667,44.0000){\scriptsize\makebox(0,0)[t]{4}}
\put(15.1667,-4.0000){\scriptsize\makebox(0,0)[b]{4}}
\put(19.5000,44.0000){\scriptsize\makebox(0,0)[t]{5}}
\put(19.5000,-4.0000){\scriptsize\makebox(0,0)[b]{5}}
\put(23.8333,44.0000){\scriptsize\makebox(0,0)[t]{6}}
\put(23.8333,-4.0000){\scriptsize\makebox(0,0)[b]{6}}
\put(28.1667,44.0000){\scriptsize\makebox(0,0)[t]{7}}
\put(28.1667,-4.0000){\scriptsize\makebox(0,0)[b]{7}}
\put(32.5000,44.0000){\scriptsize\makebox(0,0)[t]{8}}
\put(32.5000,-4.0000){\scriptsize\makebox(0,0)[b]{8}}
\put(36.8333,44.0000){\scriptsize\makebox(0,0)[t]{9}}
\put(36.8333,-4.0000){\scriptsize\makebox(0,0)[b]{9}}
\put(41.1667,44.0000){\scriptsize\makebox(0,0)[t]{10}}
\put(41.1667,-4.0000){\scriptsize\makebox(0,0)[b]{10}}
\put(45.5000,44.0000){\scriptsize\makebox(0,0)[t]{11}}
\put(45.5000,-4.0000){\scriptsize\makebox(0,0)[b]{11}}
\put(49.8333,44.0000){\scriptsize\makebox(0,0)[t]{12}}
\put(49.8333,-4.0000){\scriptsize\makebox(0,0)[b]{12}}
\put(54.1667,44.0000){\scriptsize\makebox(0,0)[t]{13}}
\put(54.1667,-4.0000){\scriptsize\makebox(0,0)[b]{13}}
\put(58.5000,44.0000){\scriptsize\makebox(0,0)[t]{14}}
\put(58.5000,-4.0000){\scriptsize\makebox(0,0)[b]{14}}
\put(62.8333,44.0000){\scriptsize\makebox(0,0)[t]{15}}
\put(62.8333,-4.0000){\scriptsize\makebox(0,0)[b]{15}}
\put(67.1667,44.0000){\scriptsize\makebox(0,0)[t]{16}}
\put(67.1667,-4.0000){\scriptsize\makebox(0,0)[b]{16}}
\put(71.5000,44.0000){\scriptsize\makebox(0,0)[t]{17}}
\put(71.5000,-4.0000){\scriptsize\makebox(0,0)[b]{17}}
\put(75.8333,44.0000){\scriptsize\makebox(0,0)[t]{18}}
\put(75.8333,-4.0000){\scriptsize\makebox(0,0)[b]{18}}
\put(80.1667,44.0000){\scriptsize\makebox(0,0)[t]{19}}
\put(80.1667,-4.0000){\scriptsize\makebox(0,0)[b]{19}}
\put(84.5000,44.0000){\scriptsize\makebox(0,0)[t]{20}}
\put(84.5000,-4.0000){\scriptsize\makebox(0,0)[b]{20}}
\put(88.8333,44.0000){\scriptsize\makebox(0,0)[t]{21}}
\put(88.8333,-4.0000){\scriptsize\makebox(0,0)[b]{21}}
\put(93.1667,44.0000){\scriptsize\makebox(0,0)[t]{22}}
\put(93.1667,-4.0000){\scriptsize\makebox(0,0)[b]{22}}
\put(97.5000,44.0000){\scriptsize\makebox(0,0)[t]{23}}
\put(97.5000,-4.0000){\scriptsize\makebox(0,0)[b]{23}}
\put(101.8333,44.0000){\scriptsize\makebox(0,0)[t]{24}}
\put(101.8333,-4.0000){\scriptsize\makebox(0,0)[b]{24}}
\put(106.1667,44.0000){\scriptsize\makebox(0,0)[t]{25}}
\put(106.1667,-4.0000){\scriptsize\makebox(0,0)[b]{25}}
\put(110.5000,44.0000){\scriptsize\makebox(0,0)[t]{26}}
\put(110.5000,-4.0000){\scriptsize\makebox(0,0)[b]{26}}
\put(114.8333,44.0000){\scriptsize\makebox(0,0)[t]{27}}
\put(114.8333,-4.0000){\scriptsize\makebox(0,0)[b]{27}}
\put(119.1667,44.0000){\scriptsize\makebox(0,0)[t]{28}}
\put(119.1667,-4.0000){\scriptsize\makebox(0,0)[b]{28}}
\put(123.5000,44.0000){\scriptsize\makebox(0,0)[t]{29}}
\put(123.5000,-4.0000){\scriptsize\makebox(0,0)[b]{29}}
\put(127.8333,44.0000){\scriptsize\makebox(0,0)[t]{30}}
\put(127.8333,-4.0000){\scriptsize\makebox(0,0)[b]{30}}
\put(-1.0000,34.5714){\line(1,0){132.0000}}
\put(-1.0000,38.0000){\line(1,0){132.0000}}
\put(-1.0000,29.1429){\line(1,0){132.0000}}
\put(-1.0000,32.5714){\line(1,0){132.0000}}
\put(-1.0000,23.7143){\line(1,0){132.0000}}
\put(-1.0000,27.1429){\line(1,0){132.0000}}
\put(-1.0000,18.2857){\line(1,0){132.0000}}
\put(-1.0000,21.7143){\line(1,0){132.0000}}
\put(-1.0000,12.8571){\line(1,0){132.0000}}
\put(-1.0000,16.2857){\line(1,0){132.0000}}
\put(-1.0000,7.4286){\line(1,0){132.0000}}
\put(-1.0000,10.8571){\line(1,0){132.0000}}
\put(-1.0000,2.0000){\line(1,0){132.0000}}
\put(-1.0000,5.4286){\line(1,0){132.0000}}
\ligneepaisse
\color{blue}
\put(-1.0000,36.2857){\color{blue}\normalsize\makebox(0,0)[r]{rq\_in\_1}}
\put(21.6667,34.5714){\line(0,1){3.4286}}
\put(65.0000,38.0000){\line(0,-1){3.4286}}
\put(82.3333,34.5714){\line(0,1){3.4286}}
\put(99.6667,38.0000){\line(0,-1){3.4286}}
\put(117.0000,34.5714){\line(0,1){3.4286}}
\put(0.0000,34.5714){\line(1,0){4.3333}}
\put(4.3333,34.5714){\line(1,0){4.3333}}
\put(8.6667,34.5714){\line(1,0){4.3333}}
\put(13.0000,34.5714){\line(1,0){4.3333}}
\put(17.3333,34.5714){\line(1,0){4.3333}}
\put(21.6667,38.0000){\line(1,0){4.3333}}
\put(26.0000,38.0000){\line(1,0){4.3333}}
\put(30.3333,38.0000){\line(1,0){4.3333}}
\put(34.6667,38.0000){\line(1,0){4.3333}}
\put(39.0000,38.0000){\line(1,0){4.3333}}
\put(43.3333,38.0000){\line(1,0){4.3333}}
\put(47.6667,38.0000){\line(1,0){4.3333}}
\put(52.0000,38.0000){\line(1,0){4.3333}}
\put(56.3333,38.0000){\line(1,0){4.3333}}
\put(60.6667,38.0000){\line(1,0){4.3333}}
\put(65.0000,34.5714){\line(1,0){4.3333}}
\put(69.3333,34.5714){\line(1,0){4.3333}}
\put(73.6667,34.5714){\line(1,0){4.3333}}
\put(78.0000,34.5714){\line(1,0){4.3333}}
\put(82.3333,38.0000){\line(1,0){4.3333}}
\put(86.6667,38.0000){\line(1,0){4.3333}}
\put(91.0000,38.0000){\line(1,0){4.3333}}
\put(95.3333,38.0000){\line(1,0){4.3333}}
\put(99.6667,34.5714){\line(1,0){4.3333}}
\put(104.0000,34.5714){\line(1,0){4.3333}}
\put(108.3333,34.5714){\line(1,0){4.3333}}
\put(112.6667,34.5714){\line(1,0){4.3333}}
\put(117.0000,38.0000){\line(1,0){4.3333}}
\put(121.3333,38.0000){\line(1,0){4.3333}}
\put(125.6667,38.0000){\line(1,0){4.3333}}
\color{blue}
\put(-1.0000,30.8571){\color{blue}\normalsize\makebox(0,0)[r]{rq\_in\_2}}
\put(0.0000,29.1429){\line(1,0){4.3333}}
\put(4.3333,29.1429){\line(1,0){4.3333}}
\put(8.6667,29.1429){\line(1,0){4.3333}}
\put(13.0000,29.1429){\line(1,0){4.3333}}
\put(17.3333,29.1429){\line(1,0){4.3333}}
\put(21.6667,29.1429){\line(1,0){4.3333}}
\put(26.0000,29.1429){\line(1,0){4.3333}}
\put(30.3333,29.1429){\line(1,0){4.3333}}
\put(34.6667,29.1429){\line(1,0){4.3333}}
\put(39.0000,29.1429){\line(1,0){4.3333}}
\put(43.3333,29.1429){\line(1,0){4.3333}}
\put(47.6667,29.1429){\line(1,0){4.3333}}
\put(52.0000,29.1429){\line(1,0){4.3333}}
\put(56.3333,29.1429){\line(1,0){4.3333}}
\put(60.6667,29.1429){\line(1,0){4.3333}}
\put(65.0000,29.1429){\line(1,0){4.3333}}
\put(69.3333,29.1429){\line(1,0){4.3333}}
\put(73.6667,29.1429){\line(1,0){4.3333}}
\put(78.0000,29.1429){\line(1,0){4.3333}}
\put(82.3333,29.1429){\line(1,0){4.3333}}
\put(86.6667,29.1429){\line(1,0){4.3333}}
\put(91.0000,29.1429){\line(1,0){4.3333}}
\put(95.3333,29.1429){\line(1,0){4.3333}}
\put(99.6667,29.1429){\line(1,0){4.3333}}
\put(104.0000,29.1429){\line(1,0){4.3333}}
\put(108.3333,29.1429){\line(1,0){4.3333}}
\put(112.6667,29.1429){\line(1,0){4.3333}}
\put(117.0000,29.1429){\line(1,0){4.3333}}
\put(121.3333,29.1429){\line(1,0){4.3333}}
\put(125.6667,29.1429){\line(1,0){4.3333}}
\color{blue}
\put(-1.0000,25.4286){\color{blue}\normalsize\makebox(0,0)[r]{rq\_in\_3}}
\put(65.0000,23.7143){\line(0,1){3.4286}}
\put(82.3333,27.1429){\line(0,-1){3.4286}}
\put(99.6667,23.7143){\line(0,1){3.4286}}
\put(117.0000,27.1429){\line(0,-1){3.4286}}
\put(0.0000,23.7143){\line(1,0){4.3333}}
\put(4.3333,23.7143){\line(1,0){4.3333}}
\put(8.6667,23.7143){\line(1,0){4.3333}}
\put(13.0000,23.7143){\line(1,0){4.3333}}
\put(17.3333,23.7143){\line(1,0){4.3333}}
\put(21.6667,23.7143){\line(1,0){4.3333}}
\put(26.0000,23.7143){\line(1,0){4.3333}}
\put(30.3333,23.7143){\line(1,0){4.3333}}
\put(34.6667,23.7143){\line(1,0){4.3333}}
\put(39.0000,23.7143){\line(1,0){4.3333}}
\put(43.3333,23.7143){\line(1,0){4.3333}}
\put(47.6667,23.7143){\line(1,0){4.3333}}
\put(52.0000,23.7143){\line(1,0){4.3333}}
\put(56.3333,23.7143){\line(1,0){4.3333}}
\put(60.6667,23.7143){\line(1,0){4.3333}}
\put(65.0000,27.1429){\line(1,0){4.3333}}
\put(69.3333,27.1429){\line(1,0){4.3333}}
\put(73.6667,27.1429){\line(1,0){4.3333}}
\put(78.0000,27.1429){\line(1,0){4.3333}}
\put(82.3333,23.7143){\line(1,0){4.3333}}
\put(86.6667,23.7143){\line(1,0){4.3333}}
\put(91.0000,23.7143){\line(1,0){4.3333}}
\put(95.3333,23.7143){\line(1,0){4.3333}}
\put(99.6667,27.1429){\line(1,0){4.3333}}
\put(104.0000,27.1429){\line(1,0){4.3333}}
\put(108.3333,27.1429){\line(1,0){4.3333}}
\put(112.6667,27.1429){\line(1,0){4.3333}}
\put(117.0000,23.7143){\line(1,0){4.3333}}
\put(121.3333,23.7143){\line(1,0){4.3333}}
\put(125.6667,23.7143){\line(1,0){4.3333}}
\color{red}
\put(-1.0000,20.0000){\color{red}\normalsize\makebox(0,0)[r]{ack\_out\_1}}
\put(21.6667,18.2857){\line(0,1){3.4286}}
\put(65.0000,21.7143){\line(0,-1){3.4286}}
\put(82.3333,18.2857){\line(0,1){3.4286}}
\put(99.6667,21.7143){\line(0,-1){3.4286}}
\put(121.3333,18.2857){\line(0,1){3.4286}}
\put(0.0000,18.2857){\line(1,0){4.3333}}
\put(4.3333,18.2857){\line(1,0){4.3333}}
\put(8.6667,18.2857){\line(1,0){4.3333}}
\put(13.0000,18.2857){\line(1,0){4.3333}}
\put(17.3333,18.2857){\line(1,0){4.3333}}
\put(21.6667,21.7143){\line(1,0){4.3333}}
\put(26.0000,21.7143){\line(1,0){4.3333}}
\put(30.3333,21.7143){\line(1,0){4.3333}}
\put(34.6667,21.7143){\line(1,0){4.3333}}
\put(39.0000,21.7143){\line(1,0){4.3333}}
\put(43.3333,21.7143){\line(1,0){4.3333}}
\put(47.6667,21.7143){\line(1,0){4.3333}}
\put(52.0000,21.7143){\line(1,0){4.3333}}
\put(56.3333,21.7143){\line(1,0){4.3333}}
\put(60.6667,21.7143){\line(1,0){4.3333}}
\put(65.0000,18.2857){\line(1,0){4.3333}}
\put(69.3333,18.2857){\line(1,0){4.3333}}
\put(73.6667,18.2857){\line(1,0){4.3333}}
\put(78.0000,18.2857){\line(1,0){4.3333}}
\put(82.3333,21.7143){\line(1,0){4.3333}}
\put(86.6667,21.7143){\line(1,0){4.3333}}
\put(91.0000,21.7143){\line(1,0){4.3333}}
\put(95.3333,21.7143){\line(1,0){4.3333}}
\put(99.6667,18.2857){\line(1,0){4.3333}}
\put(104.0000,18.2857){\line(1,0){4.3333}}
\put(108.3333,18.2857){\line(1,0){4.3333}}
\put(112.6667,18.2857){\line(1,0){4.3333}}
\put(117.0000,18.2857){\line(1,0){4.3333}}
\put(121.3333,21.7143){\line(1,0){4.3333}}
\put(125.6667,21.7143){\line(1,0){4.3333}}
\color{red}
\put(-1.0000,14.5714){\color{red}\normalsize\makebox(0,0)[r]{ack\_out\_2}}
\put(0.0000,12.8571){\line(1,0){4.3333}}
\put(4.3333,12.8571){\line(1,0){4.3333}}
\put(8.6667,12.8571){\line(1,0){4.3333}}
\put(13.0000,12.8571){\line(1,0){4.3333}}
\put(17.3333,12.8571){\line(1,0){4.3333}}
\put(21.6667,12.8571){\line(1,0){4.3333}}
\put(26.0000,12.8571){\line(1,0){4.3333}}
\put(30.3333,12.8571){\line(1,0){4.3333}}
\put(34.6667,12.8571){\line(1,0){4.3333}}
\put(39.0000,12.8571){\line(1,0){4.3333}}
\put(43.3333,12.8571){\line(1,0){4.3333}}
\put(47.6667,12.8571){\line(1,0){4.3333}}
\put(52.0000,12.8571){\line(1,0){4.3333}}
\put(56.3333,12.8571){\line(1,0){4.3333}}
\put(60.6667,12.8571){\line(1,0){4.3333}}
\put(65.0000,12.8571){\line(1,0){4.3333}}
\put(69.3333,12.8571){\line(1,0){4.3333}}
\put(73.6667,12.8571){\line(1,0){4.3333}}
\put(78.0000,12.8571){\line(1,0){4.3333}}
\put(82.3333,12.8571){\line(1,0){4.3333}}
\put(86.6667,12.8571){\line(1,0){4.3333}}
\put(91.0000,12.8571){\line(1,0){4.3333}}
\put(95.3333,12.8571){\line(1,0){4.3333}}
\put(99.6667,12.8571){\line(1,0){4.3333}}
\put(104.0000,12.8571){\line(1,0){4.3333}}
\put(108.3333,12.8571){\line(1,0){4.3333}}
\put(112.6667,12.8571){\line(1,0){4.3333}}
\put(117.0000,12.8571){\line(1,0){4.3333}}
\put(121.3333,12.8571){\line(1,0){4.3333}}
\put(125.6667,12.8571){\line(1,0){4.3333}}
\color{red}
\put(-1.0000,9.1429){\color{red}\normalsize\makebox(0,0)[r]{ack\_out\_3}}
\put(65.0000,7.4286){\line(0,1){3.4286}}
\put(82.3333,10.8571){\line(0,-1){3.4286}}
\put(99.6667,7.4286){\line(0,1){3.4286}}
\put(117.0000,10.8571){\line(0,-1){3.4286}}
\put(0.0000,7.4286){\line(1,0){4.3333}}
\put(4.3333,7.4286){\line(1,0){4.3333}}
\put(8.6667,7.4286){\line(1,0){4.3333}}
\put(13.0000,7.4286){\line(1,0){4.3333}}
\put(17.3333,7.4286){\line(1,0){4.3333}}
\put(21.6667,7.4286){\line(1,0){4.3333}}
\put(26.0000,7.4286){\line(1,0){4.3333}}
\put(30.3333,7.4286){\line(1,0){4.3333}}
\put(34.6667,7.4286){\line(1,0){4.3333}}
\put(39.0000,7.4286){\line(1,0){4.3333}}
\put(43.3333,7.4286){\line(1,0){4.3333}}
\put(47.6667,7.4286){\line(1,0){4.3333}}
\put(52.0000,7.4286){\line(1,0){4.3333}}
\put(56.3333,7.4286){\line(1,0){4.3333}}
\put(60.6667,7.4286){\line(1,0){4.3333}}
\put(65.0000,10.8571){\line(1,0){4.3333}}
\put(69.3333,10.8571){\line(1,0){4.3333}}
\put(73.6667,10.8571){\line(1,0){4.3333}}
\put(78.0000,10.8571){\line(1,0){4.3333}}
\put(82.3333,7.4286){\line(1,0){4.3333}}
\put(86.6667,7.4286){\line(1,0){4.3333}}
\put(91.0000,7.4286){\line(1,0){4.3333}}
\put(95.3333,7.4286){\line(1,0){4.3333}}
\put(99.6667,10.8571){\line(1,0){4.3333}}
\put(104.0000,10.8571){\line(1,0){4.3333}}
\put(108.3333,10.8571){\line(1,0){4.3333}}
\put(112.6667,10.8571){\line(1,0){4.3333}}
\put(117.0000,7.4286){\line(1,0){4.3333}}
\put(121.3333,7.4286){\line(1,0){4.3333}}
\put(125.6667,7.4286){\line(1,0){4.3333}}
\color{red}
\put(-1.0000,3.7143){\color{red}\normalsize\makebox(0,0)[r]{gr\_out}}
\put(21.6667,5.4286){\line(0,-1){3.4286}}
\put(0.0000,5.4286){\line(1,0){4.3333}}
\put(4.3333,5.4286){\line(1,0){4.3333}}
\put(8.6667,5.4286){\line(1,0){4.3333}}
\put(13.0000,5.4286){\line(1,0){4.3333}}
\put(17.3333,5.4286){\line(1,0){4.3333}}
\put(21.6667,2.0000){\line(1,0){4.3333}}
\put(26.0000,2.0000){\line(1,0){4.3333}}
\put(30.3333,2.0000){\line(1,0){4.3333}}
\put(34.6667,2.0000){\line(1,0){4.3333}}
\put(39.0000,2.0000){\line(1,0){4.3333}}
\put(43.3333,2.0000){\line(1,0){4.3333}}
\put(47.6667,2.0000){\line(1,0){4.3333}}
\put(52.0000,2.0000){\line(1,0){4.3333}}
\put(56.3333,2.0000){\line(1,0){4.3333}}
\put(60.6667,2.0000){\line(1,0){4.3333}}
\put(65.0000,2.0000){\line(1,0){4.3333}}
\put(69.3333,2.0000){\line(1,0){4.3333}}
\put(73.6667,2.0000){\line(1,0){4.3333}}
\put(78.0000,2.0000){\line(1,0){4.3333}}
\put(82.3333,2.0000){\line(1,0){4.3333}}
\put(86.6667,2.0000){\line(1,0){4.3333}}
\put(91.0000,2.0000){\line(1,0){4.3333}}
\put(95.3333,2.0000){\line(1,0){4.3333}}
\put(99.6667,2.0000){\line(1,0){4.3333}}
\put(104.0000,2.0000){\line(1,0){4.3333}}
\put(108.3333,2.0000){\line(1,0){4.3333}}
\put(112.6667,2.0000){\line(1,0){4.3333}}
\put(117.0000,2.0000){\line(1,0){4.3333}}
\put(121.3333,2.0000){\line(1,0){4.3333}}
\put(125.6667,2.0000){\line(1,0){4.3333}}
\end{picture}
\\
\vspace{10mm}
\noindent

\setlength{\unitlength}{1mm}
\begin{picture}(150.0000,50.0000)(-19.0000,-5.0000)
\fboxsep 0pt
\lignefine
\color{black}
\multiput(0.0000,-5.0000)(4.3333,0){31}{\line(0,1){50.0000}}
\put(2.1667,44.0000){\scriptsize\makebox(0,0)[t]{31}}
\put(2.1667,-4.0000){\scriptsize\makebox(0,0)[b]{31}}
\put(6.5000,44.0000){\scriptsize\makebox(0,0)[t]{32}}
\put(6.5000,-4.0000){\scriptsize\makebox(0,0)[b]{32}}
\put(10.8333,44.0000){\scriptsize\makebox(0,0)[t]{33}}
\put(10.8333,-4.0000){\scriptsize\makebox(0,0)[b]{33}}
\put(15.1667,44.0000){\scriptsize\makebox(0,0)[t]{34}}
\put(15.1667,-4.0000){\scriptsize\makebox(0,0)[b]{34}}
\put(19.5000,44.0000){\scriptsize\makebox(0,0)[t]{35}}
\put(19.5000,-4.0000){\scriptsize\makebox(0,0)[b]{35}}
\put(23.8333,44.0000){\scriptsize\makebox(0,0)[t]{36}}
\put(23.8333,-4.0000){\scriptsize\makebox(0,0)[b]{36}}
\put(28.1667,44.0000){\scriptsize\makebox(0,0)[t]{37}}
\put(28.1667,-4.0000){\scriptsize\makebox(0,0)[b]{37}}
\put(32.5000,44.0000){\scriptsize\makebox(0,0)[t]{38}}
\put(32.5000,-4.0000){\scriptsize\makebox(0,0)[b]{38}}
\put(36.8333,44.0000){\scriptsize\makebox(0,0)[t]{39}}
\put(36.8333,-4.0000){\scriptsize\makebox(0,0)[b]{39}}
\put(41.1667,44.0000){\scriptsize\makebox(0,0)[t]{40}}
\put(41.1667,-4.0000){\scriptsize\makebox(0,0)[b]{40}}
\put(45.5000,44.0000){\scriptsize\makebox(0,0)[t]{41}}
\put(45.5000,-4.0000){\scriptsize\makebox(0,0)[b]{41}}
\put(49.8333,44.0000){\scriptsize\makebox(0,0)[t]{42}}
\put(49.8333,-4.0000){\scriptsize\makebox(0,0)[b]{42}}
\put(54.1667,44.0000){\scriptsize\makebox(0,0)[t]{43}}
\put(54.1667,-4.0000){\scriptsize\makebox(0,0)[b]{43}}
\put(58.5000,44.0000){\scriptsize\makebox(0,0)[t]{44}}
\put(58.5000,-4.0000){\scriptsize\makebox(0,0)[b]{44}}
\put(62.8333,44.0000){\scriptsize\makebox(0,0)[t]{45}}
\put(62.8333,-4.0000){\scriptsize\makebox(0,0)[b]{45}}
\put(67.1667,44.0000){\scriptsize\makebox(0,0)[t]{46}}
\put(67.1667,-4.0000){\scriptsize\makebox(0,0)[b]{46}}
\put(71.5000,44.0000){\scriptsize\makebox(0,0)[t]{47}}
\put(71.5000,-4.0000){\scriptsize\makebox(0,0)[b]{47}}
\put(75.8333,44.0000){\scriptsize\makebox(0,0)[t]{48}}
\put(75.8333,-4.0000){\scriptsize\makebox(0,0)[b]{48}}
\put(80.1667,44.0000){\scriptsize\makebox(0,0)[t]{49}}
\put(80.1667,-4.0000){\scriptsize\makebox(0,0)[b]{49}}
\put(84.5000,44.0000){\scriptsize\makebox(0,0)[t]{50}}
\put(84.5000,-4.0000){\scriptsize\makebox(0,0)[b]{50}}
\put(88.8333,44.0000){\scriptsize\makebox(0,0)[t]{51}}
\put(88.8333,-4.0000){\scriptsize\makebox(0,0)[b]{51}}
\put(93.1667,44.0000){\scriptsize\makebox(0,0)[t]{52}}
\put(93.1667,-4.0000){\scriptsize\makebox(0,0)[b]{52}}
\put(97.5000,44.0000){\scriptsize\makebox(0,0)[t]{53}}
\put(97.5000,-4.0000){\scriptsize\makebox(0,0)[b]{53}}
\put(101.8333,44.0000){\scriptsize\makebox(0,0)[t]{54}}
\put(101.8333,-4.0000){\scriptsize\makebox(0,0)[b]{54}}
\put(106.1667,44.0000){\scriptsize\makebox(0,0)[t]{55}}
\put(106.1667,-4.0000){\scriptsize\makebox(0,0)[b]{55}}
\put(110.5000,44.0000){\scriptsize\makebox(0,0)[t]{56}}
\put(110.5000,-4.0000){\scriptsize\makebox(0,0)[b]{56}}
\put(114.8333,44.0000){\scriptsize\makebox(0,0)[t]{57}}
\put(114.8333,-4.0000){\scriptsize\makebox(0,0)[b]{57}}
\put(119.1667,44.0000){\scriptsize\makebox(0,0)[t]{58}}
\put(119.1667,-4.0000){\scriptsize\makebox(0,0)[b]{58}}
\put(123.5000,44.0000){\scriptsize\makebox(0,0)[t]{59}}
\put(123.5000,-4.0000){\scriptsize\makebox(0,0)[b]{59}}
\put(127.8333,44.0000){\scriptsize\makebox(0,0)[t]{60}}
\put(127.8333,-4.0000){\scriptsize\makebox(0,0)[b]{60}}
\put(-1.0000,34.5714){\line(1,0){132.0000}}
\put(-1.0000,38.0000){\line(1,0){132.0000}}
\put(-1.0000,29.1429){\line(1,0){132.0000}}
\put(-1.0000,32.5714){\line(1,0){132.0000}}
\put(-1.0000,23.7143){\line(1,0){132.0000}}
\put(-1.0000,27.1429){\line(1,0){132.0000}}
\put(-1.0000,18.2857){\line(1,0){132.0000}}
\put(-1.0000,21.7143){\line(1,0){132.0000}}
\put(-1.0000,12.8571){\line(1,0){132.0000}}
\put(-1.0000,16.2857){\line(1,0){132.0000}}
\put(-1.0000,7.4286){\line(1,0){132.0000}}
\put(-1.0000,10.8571){\line(1,0){132.0000}}
\put(-1.0000,2.0000){\line(1,0){132.0000}}
\put(-1.0000,5.4286){\line(1,0){132.0000}}
\ligneepaisse
\color{blue}
\put(-1.0000,36.2857){\color{blue}\normalsize\makebox(0,0)[r]{rq\_in\_1}}
\put(47.6667,38.0000){\line(0,-1){3.4286}}
\put(56.3333,34.5714){\line(0,1){3.4286}}
\put(0.0000,38.0000){\line(1,0){4.3333}}
\put(4.3333,38.0000){\line(1,0){4.3333}}
\put(8.6667,38.0000){\line(1,0){4.3333}}
\put(13.0000,38.0000){\line(1,0){4.3333}}
\put(17.3333,38.0000){\line(1,0){4.3333}}
\put(21.6667,38.0000){\line(1,0){4.3333}}
\put(26.0000,38.0000){\line(1,0){4.3333}}
\put(30.3333,38.0000){\line(1,0){4.3333}}
\put(34.6667,38.0000){\line(1,0){4.3333}}
\put(39.0000,38.0000){\line(1,0){4.3333}}
\put(43.3333,38.0000){\line(1,0){4.3333}}
\put(47.6667,34.5714){\line(1,0){4.3333}}
\put(52.0000,34.5714){\line(1,0){4.3333}}
\put(56.3333,38.0000){\line(1,0){4.3333}}
\put(60.6667,38.0000){\line(1,0){4.3333}}
\put(65.0000,38.0000){\line(1,0){4.3333}}
\put(69.3333,38.0000){\line(1,0){4.3333}}
\put(73.6667,38.0000){\line(1,0){4.3333}}
\put(78.0000,38.0000){\line(1,0){4.3333}}
\put(82.3333,38.0000){\line(1,0){4.3333}}
\put(86.6667,38.0000){\line(1,0){4.3333}}
\put(91.0000,38.0000){\line(1,0){4.3333}}
\put(95.3333,38.0000){\line(1,0){4.3333}}
\put(99.6667,38.0000){\line(1,0){4.3333}}
\put(104.0000,38.0000){\line(1,0){4.3333}}
\put(108.3333,38.0000){\line(1,0){4.3333}}
\put(112.6667,38.0000){\line(1,0){4.3333}}
\put(117.0000,38.0000){\line(1,0){4.3333}}
\put(121.3333,38.0000){\line(1,0){4.3333}}
\put(125.6667,38.0000){\line(1,0){4.3333}}
\color{blue}
\put(-1.0000,30.8571){\color{blue}\normalsize\makebox(0,0)[r]{rq\_in\_2}}
\put(56.3333,29.1429){\line(0,1){3.4286}}
\put(0.0000,29.1429){\line(1,0){4.3333}}
\put(4.3333,29.1429){\line(1,0){4.3333}}
\put(8.6667,29.1429){\line(1,0){4.3333}}
\put(13.0000,29.1429){\line(1,0){4.3333}}
\put(17.3333,29.1429){\line(1,0){4.3333}}
\put(21.6667,29.1429){\line(1,0){4.3333}}
\put(26.0000,29.1429){\line(1,0){4.3333}}
\put(30.3333,29.1429){\line(1,0){4.3333}}
\put(34.6667,29.1429){\line(1,0){4.3333}}
\put(39.0000,29.1429){\line(1,0){4.3333}}
\put(43.3333,29.1429){\line(1,0){4.3333}}
\put(47.6667,29.1429){\line(1,0){4.3333}}
\put(52.0000,29.1429){\line(1,0){4.3333}}
\put(56.3333,32.5714){\line(1,0){4.3333}}
\put(60.6667,32.5714){\line(1,0){4.3333}}
\put(65.0000,32.5714){\line(1,0){4.3333}}
\put(69.3333,32.5714){\line(1,0){4.3333}}
\put(73.6667,32.5714){\line(1,0){4.3333}}
\put(78.0000,32.5714){\line(1,0){4.3333}}
\put(82.3333,32.5714){\line(1,0){4.3333}}
\put(86.6667,32.5714){\line(1,0){4.3333}}
\put(91.0000,32.5714){\line(1,0){4.3333}}
\put(95.3333,32.5714){\line(1,0){4.3333}}
\put(99.6667,32.5714){\line(1,0){4.3333}}
\put(104.0000,32.5714){\line(1,0){4.3333}}
\put(108.3333,32.5714){\line(1,0){4.3333}}
\put(112.6667,32.5714){\line(1,0){4.3333}}
\put(117.0000,32.5714){\line(1,0){4.3333}}
\put(121.3333,32.5714){\line(1,0){4.3333}}
\put(125.6667,32.5714){\line(1,0){4.3333}}
\color{blue}
\put(-1.0000,25.4286){\color{blue}\normalsize\makebox(0,0)[r]{rq\_in\_3}}
\put(0.0000,23.7143){\line(0,1){3.4286}}
\put(47.6667,27.1429){\line(0,-1){3.4286}}
\put(0.0000,27.1429){\line(1,0){4.3333}}
\put(4.3333,27.1429){\line(1,0){4.3333}}
\put(8.6667,27.1429){\line(1,0){4.3333}}
\put(13.0000,27.1429){\line(1,0){4.3333}}
\put(17.3333,27.1429){\line(1,0){4.3333}}
\put(21.6667,27.1429){\line(1,0){4.3333}}
\put(26.0000,27.1429){\line(1,0){4.3333}}
\put(30.3333,27.1429){\line(1,0){4.3333}}
\put(34.6667,27.1429){\line(1,0){4.3333}}
\put(39.0000,27.1429){\line(1,0){4.3333}}
\put(43.3333,27.1429){\line(1,0){4.3333}}
\put(47.6667,23.7143){\line(1,0){4.3333}}
\put(52.0000,23.7143){\line(1,0){4.3333}}
\put(56.3333,23.7143){\line(1,0){4.3333}}
\put(60.6667,23.7143){\line(1,0){4.3333}}
\put(65.0000,23.7143){\line(1,0){4.3333}}
\put(69.3333,23.7143){\line(1,0){4.3333}}
\put(73.6667,23.7143){\line(1,0){4.3333}}
\put(78.0000,23.7143){\line(1,0){4.3333}}
\put(82.3333,23.7143){\line(1,0){4.3333}}
\put(86.6667,23.7143){\line(1,0){4.3333}}
\put(91.0000,23.7143){\line(1,0){4.3333}}
\put(95.3333,23.7143){\line(1,0){4.3333}}
\put(99.6667,23.7143){\line(1,0){4.3333}}
\put(104.0000,23.7143){\line(1,0){4.3333}}
\put(108.3333,23.7143){\line(1,0){4.3333}}
\put(112.6667,23.7143){\line(1,0){4.3333}}
\put(117.0000,23.7143){\line(1,0){4.3333}}
\put(121.3333,23.7143){\line(1,0){4.3333}}
\put(125.6667,23.7143){\line(1,0){4.3333}}
\color{red}
\put(-1.0000,20.0000){\color{red}\normalsize\makebox(0,0)[r]{ack\_out\_1}}
\put(0.0000,21.7143){\line(0,-1){3.4286}}
\put(4.3333,18.2857){\line(0,1){3.4286}}
\put(8.6667,21.7143){\line(0,-1){3.4286}}
\put(17.3333,18.2857){\line(0,1){3.4286}}
\put(21.6667,21.7143){\line(0,-1){3.4286}}
\put(30.3333,18.2857){\line(0,1){3.4286}}
\put(34.6667,21.7143){\line(0,-1){3.4286}}
\put(43.3333,18.2857){\line(0,1){3.4286}}
\put(47.6667,21.7143){\line(0,-1){3.4286}}
\put(69.3333,18.2857){\line(0,1){3.4286}}
\put(73.6667,21.7143){\line(0,-1){3.4286}}
\put(82.3333,18.2857){\line(0,1){3.4286}}
\put(86.6667,21.7143){\line(0,-1){3.4286}}
\put(95.3333,18.2857){\line(0,1){3.4286}}
\put(99.6667,21.7143){\line(0,-1){3.4286}}
\put(108.3333,18.2857){\line(0,1){3.4286}}
\put(112.6667,21.7143){\line(0,-1){3.4286}}
\put(121.3333,18.2857){\line(0,1){3.4286}}
\put(125.6667,21.7143){\line(0,-1){3.4286}}
\put(0.0000,18.2857){\line(1,0){4.3333}}
\put(4.3333,21.7143){\line(1,0){4.3333}}
\put(8.6667,18.2857){\line(1,0){4.3333}}
\put(13.0000,18.2857){\line(1,0){4.3333}}
\put(17.3333,21.7143){\line(1,0){4.3333}}
\put(21.6667,18.2857){\line(1,0){4.3333}}
\put(26.0000,18.2857){\line(1,0){4.3333}}
\put(30.3333,21.7143){\line(1,0){4.3333}}
\put(34.6667,18.2857){\line(1,0){4.3333}}
\put(39.0000,18.2857){\line(1,0){4.3333}}
\put(43.3333,21.7143){\line(1,0){4.3333}}
\put(47.6667,18.2857){\line(1,0){4.3333}}
\put(52.0000,18.2857){\line(1,0){4.3333}}
\put(56.3333,18.2857){\line(1,0){4.3333}}
\put(60.6667,18.2857){\line(1,0){4.3333}}
\put(65.0000,18.2857){\line(1,0){4.3333}}
\put(69.3333,21.7143){\line(1,0){4.3333}}
\put(73.6667,18.2857){\line(1,0){4.3333}}
\put(78.0000,18.2857){\line(1,0){4.3333}}
\put(82.3333,21.7143){\line(1,0){4.3333}}
\put(86.6667,18.2857){\line(1,0){4.3333}}
\put(91.0000,18.2857){\line(1,0){4.3333}}
\put(95.3333,21.7143){\line(1,0){4.3333}}
\put(99.6667,18.2857){\line(1,0){4.3333}}
\put(104.0000,18.2857){\line(1,0){4.3333}}
\put(108.3333,21.7143){\line(1,0){4.3333}}
\put(112.6667,18.2857){\line(1,0){4.3333}}
\put(117.0000,18.2857){\line(1,0){4.3333}}
\put(121.3333,21.7143){\line(1,0){4.3333}}
\put(125.6667,18.2857){\line(1,0){4.3333}}
\color{red}
\put(-1.0000,14.5714){\color{red}\normalsize\makebox(0,0)[r]{ack\_out\_2}}
\put(56.3333,12.8571){\line(0,1){3.4286}}
\put(69.3333,16.2857){\line(0,-1){3.4286}}
\put(73.6667,12.8571){\line(0,1){3.4286}}
\put(82.3333,16.2857){\line(0,-1){3.4286}}
\put(86.6667,12.8571){\line(0,1){3.4286}}
\put(95.3333,16.2857){\line(0,-1){3.4286}}
\put(99.6667,12.8571){\line(0,1){3.4286}}
\put(108.3333,16.2857){\line(0,-1){3.4286}}
\put(112.6667,12.8571){\line(0,1){3.4286}}
\put(121.3333,16.2857){\line(0,-1){3.4286}}
\put(125.6667,12.8571){\line(0,1){3.4286}}
\put(0.0000,12.8571){\line(1,0){4.3333}}
\put(4.3333,12.8571){\line(1,0){4.3333}}
\put(8.6667,12.8571){\line(1,0){4.3333}}
\put(13.0000,12.8571){\line(1,0){4.3333}}
\put(17.3333,12.8571){\line(1,0){4.3333}}
\put(21.6667,12.8571){\line(1,0){4.3333}}
\put(26.0000,12.8571){\line(1,0){4.3333}}
\put(30.3333,12.8571){\line(1,0){4.3333}}
\put(34.6667,12.8571){\line(1,0){4.3333}}
\put(39.0000,12.8571){\line(1,0){4.3333}}
\put(43.3333,12.8571){\line(1,0){4.3333}}
\put(47.6667,12.8571){\line(1,0){4.3333}}
\put(52.0000,12.8571){\line(1,0){4.3333}}
\put(56.3333,16.2857){\line(1,0){4.3333}}
\put(60.6667,16.2857){\line(1,0){4.3333}}
\put(65.0000,16.2857){\line(1,0){4.3333}}
\put(69.3333,12.8571){\line(1,0){4.3333}}
\put(73.6667,16.2857){\line(1,0){4.3333}}
\put(78.0000,16.2857){\line(1,0){4.3333}}
\put(82.3333,12.8571){\line(1,0){4.3333}}
\put(86.6667,16.2857){\line(1,0){4.3333}}
\put(91.0000,16.2857){\line(1,0){4.3333}}
\put(95.3333,12.8571){\line(1,0){4.3333}}
\put(99.6667,16.2857){\line(1,0){4.3333}}
\put(104.0000,16.2857){\line(1,0){4.3333}}
\put(108.3333,12.8571){\line(1,0){4.3333}}
\put(112.6667,16.2857){\line(1,0){4.3333}}
\put(117.0000,16.2857){\line(1,0){4.3333}}
\put(121.3333,12.8571){\line(1,0){4.3333}}
\put(125.6667,16.2857){\line(1,0){4.3333}}
\color{red}
\put(-1.0000,9.1429){\color{red}\normalsize\makebox(0,0)[r]{ack\_out\_3}}
\put(0.0000,7.4286){\line(0,1){3.4286}}
\put(4.3333,10.8571){\line(0,-1){3.4286}}
\put(8.6667,7.4286){\line(0,1){3.4286}}
\put(17.3333,10.8571){\line(0,-1){3.4286}}
\put(21.6667,7.4286){\line(0,1){3.4286}}
\put(30.3333,10.8571){\line(0,-1){3.4286}}
\put(34.6667,7.4286){\line(0,1){3.4286}}
\put(43.3333,10.8571){\line(0,-1){3.4286}}
\put(0.0000,10.8571){\line(1,0){4.3333}}
\put(4.3333,7.4286){\line(1,0){4.3333}}
\put(8.6667,10.8571){\line(1,0){4.3333}}
\put(13.0000,10.8571){\line(1,0){4.3333}}
\put(17.3333,7.4286){\line(1,0){4.3333}}
\put(21.6667,10.8571){\line(1,0){4.3333}}
\put(26.0000,10.8571){\line(1,0){4.3333}}
\put(30.3333,7.4286){\line(1,0){4.3333}}
\put(34.6667,10.8571){\line(1,0){4.3333}}
\put(39.0000,10.8571){\line(1,0){4.3333}}
\put(43.3333,7.4286){\line(1,0){4.3333}}
\put(47.6667,7.4286){\line(1,0){4.3333}}
\put(52.0000,7.4286){\line(1,0){4.3333}}
\put(56.3333,7.4286){\line(1,0){4.3333}}
\put(60.6667,7.4286){\line(1,0){4.3333}}
\put(65.0000,7.4286){\line(1,0){4.3333}}
\put(69.3333,7.4286){\line(1,0){4.3333}}
\put(73.6667,7.4286){\line(1,0){4.3333}}
\put(78.0000,7.4286){\line(1,0){4.3333}}
\put(82.3333,7.4286){\line(1,0){4.3333}}
\put(86.6667,7.4286){\line(1,0){4.3333}}
\put(91.0000,7.4286){\line(1,0){4.3333}}
\put(95.3333,7.4286){\line(1,0){4.3333}}
\put(99.6667,7.4286){\line(1,0){4.3333}}
\put(104.0000,7.4286){\line(1,0){4.3333}}
\put(108.3333,7.4286){\line(1,0){4.3333}}
\put(112.6667,7.4286){\line(1,0){4.3333}}
\put(117.0000,7.4286){\line(1,0){4.3333}}
\put(121.3333,7.4286){\line(1,0){4.3333}}
\put(125.6667,7.4286){\line(1,0){4.3333}}
\color{red}
\put(-1.0000,3.7143){\color{red}\normalsize\makebox(0,0)[r]{gr\_out}}
\put(47.6667,2.0000){\line(0,1){3.4286}}
\put(56.3333,5.4286){\line(0,-1){3.4286}}
\put(0.0000,2.0000){\line(1,0){4.3333}}
\put(4.3333,2.0000){\line(1,0){4.3333}}
\put(8.6667,2.0000){\line(1,0){4.3333}}
\put(13.0000,2.0000){\line(1,0){4.3333}}
\put(17.3333,2.0000){\line(1,0){4.3333}}
\put(21.6667,2.0000){\line(1,0){4.3333}}
\put(26.0000,2.0000){\line(1,0){4.3333}}
\put(30.3333,2.0000){\line(1,0){4.3333}}
\put(34.6667,2.0000){\line(1,0){4.3333}}
\put(39.0000,2.0000){\line(1,0){4.3333}}
\put(43.3333,2.0000){\line(1,0){4.3333}}
\put(47.6667,5.4286){\line(1,0){4.3333}}
\put(52.0000,5.4286){\line(1,0){4.3333}}
\put(56.3333,2.0000){\line(1,0){4.3333}}
\put(60.6667,2.0000){\line(1,0){4.3333}}
\put(65.0000,2.0000){\line(1,0){4.3333}}
\put(69.3333,2.0000){\line(1,0){4.3333}}
\put(73.6667,2.0000){\line(1,0){4.3333}}
\put(78.0000,2.0000){\line(1,0){4.3333}}
\put(82.3333,2.0000){\line(1,0){4.3333}}
\put(86.6667,2.0000){\line(1,0){4.3333}}
\put(91.0000,2.0000){\line(1,0){4.3333}}
\put(95.3333,2.0000){\line(1,0){4.3333}}
\put(99.6667,2.0000){\line(1,0){4.3333}}
\put(104.0000,2.0000){\line(1,0){4.3333}}
\put(108.3333,2.0000){\line(1,0){4.3333}}
\put(112.6667,2.0000){\line(1,0){4.3333}}
\put(117.0000,2.0000){\line(1,0){4.3333}}
\put(121.3333,2.0000){\line(1,0){4.3333}}
\put(125.6667,2.0000){\line(1,0){4.3333}}
\end{picture}
\\
\vspace{10mm}

\noindent
\setlength{\unitlength}{1mm}
\begin{picture}(150.0000,50.0000)(-19.0000,-5.0000)
\fboxsep 0pt
\lignefine
\color{black}
\multiput(0.0000,-5.0000)(4.3333,0){31}{\line(0,1){50.0000}}
\put(2.1667,44.0000){\scriptsize\makebox(0,0)[t]{61}}
\put(2.1667,-4.0000){\scriptsize\makebox(0,0)[b]{61}}
\put(6.5000,44.0000){\scriptsize\makebox(0,0)[t]{62}}
\put(6.5000,-4.0000){\scriptsize\makebox(0,0)[b]{62}}
\put(10.8333,44.0000){\scriptsize\makebox(0,0)[t]{63}}
\put(10.8333,-4.0000){\scriptsize\makebox(0,0)[b]{63}}
\put(15.1667,44.0000){\scriptsize\makebox(0,0)[t]{64}}
\put(15.1667,-4.0000){\scriptsize\makebox(0,0)[b]{64}}
\put(19.5000,44.0000){\scriptsize\makebox(0,0)[t]{65}}
\put(19.5000,-4.0000){\scriptsize\makebox(0,0)[b]{65}}
\put(23.8333,44.0000){\scriptsize\makebox(0,0)[t]{66}}
\put(23.8333,-4.0000){\scriptsize\makebox(0,0)[b]{66}}
\put(28.1667,44.0000){\scriptsize\makebox(0,0)[t]{67}}
\put(28.1667,-4.0000){\scriptsize\makebox(0,0)[b]{67}}
\put(32.5000,44.0000){\scriptsize\makebox(0,0)[t]{68}}
\put(32.5000,-4.0000){\scriptsize\makebox(0,0)[b]{68}}
\put(36.8333,44.0000){\scriptsize\makebox(0,0)[t]{69}}
\put(36.8333,-4.0000){\scriptsize\makebox(0,0)[b]{69}}
\put(41.1667,44.0000){\scriptsize\makebox(0,0)[t]{70}}
\put(41.1667,-4.0000){\scriptsize\makebox(0,0)[b]{70}}
\put(45.5000,44.0000){\scriptsize\makebox(0,0)[t]{71}}
\put(45.5000,-4.0000){\scriptsize\makebox(0,0)[b]{71}}
\put(49.8333,44.0000){\scriptsize\makebox(0,0)[t]{72}}
\put(49.8333,-4.0000){\scriptsize\makebox(0,0)[b]{72}}
\put(54.1667,44.0000){\scriptsize\makebox(0,0)[t]{73}}
\put(54.1667,-4.0000){\scriptsize\makebox(0,0)[b]{73}}
\put(58.5000,44.0000){\scriptsize\makebox(0,0)[t]{74}}
\put(58.5000,-4.0000){\scriptsize\makebox(0,0)[b]{74}}
\put(62.8333,44.0000){\scriptsize\makebox(0,0)[t]{75}}
\put(62.8333,-4.0000){\scriptsize\makebox(0,0)[b]{75}}
\put(67.1667,44.0000){\scriptsize\makebox(0,0)[t]{76}}
\put(67.1667,-4.0000){\scriptsize\makebox(0,0)[b]{76}}
\put(71.5000,44.0000){\scriptsize\makebox(0,0)[t]{77}}
\put(71.5000,-4.0000){\scriptsize\makebox(0,0)[b]{77}}
\put(75.8333,44.0000){\scriptsize\makebox(0,0)[t]{78}}
\put(75.8333,-4.0000){\scriptsize\makebox(0,0)[b]{78}}
\put(80.1667,44.0000){\scriptsize\makebox(0,0)[t]{79}}
\put(80.1667,-4.0000){\scriptsize\makebox(0,0)[b]{79}}
\put(84.5000,44.0000){\scriptsize\makebox(0,0)[t]{80}}
\put(84.5000,-4.0000){\scriptsize\makebox(0,0)[b]{80}}
\put(88.8333,44.0000){\scriptsize\makebox(0,0)[t]{81}}
\put(88.8333,-4.0000){\scriptsize\makebox(0,0)[b]{81}}
\put(93.1667,44.0000){\scriptsize\makebox(0,0)[t]{82}}
\put(93.1667,-4.0000){\scriptsize\makebox(0,0)[b]{82}}
\put(97.5000,44.0000){\scriptsize\makebox(0,0)[t]{83}}
\put(97.5000,-4.0000){\scriptsize\makebox(0,0)[b]{83}}
\put(101.8333,44.0000){\scriptsize\makebox(0,0)[t]{84}}
\put(101.8333,-4.0000){\scriptsize\makebox(0,0)[b]{84}}
\put(106.1667,44.0000){\scriptsize\makebox(0,0)[t]{85}}
\put(106.1667,-4.0000){\scriptsize\makebox(0,0)[b]{85}}
\put(110.5000,44.0000){\scriptsize\makebox(0,0)[t]{86}}
\put(110.5000,-4.0000){\scriptsize\makebox(0,0)[b]{86}}
\put(114.8333,44.0000){\scriptsize\makebox(0,0)[t]{87}}
\put(114.8333,-4.0000){\scriptsize\makebox(0,0)[b]{87}}
\put(119.1667,44.0000){\scriptsize\makebox(0,0)[t]{88}}
\put(119.1667,-4.0000){\scriptsize\makebox(0,0)[b]{88}}
\put(123.5000,44.0000){\scriptsize\makebox(0,0)[t]{89}}
\put(123.5000,-4.0000){\scriptsize\makebox(0,0)[b]{89}}
\put(127.8333,44.0000){\scriptsize\makebox(0,0)[t]{90}}
\put(127.8333,-4.0000){\scriptsize\makebox(0,0)[b]{90}}
\put(-1.0000,34.5714){\line(1,0){132.0000}}
\put(-1.0000,38.0000){\line(1,0){132.0000}}
\put(-1.0000,29.1429){\line(1,0){132.0000}}
\put(-1.0000,32.5714){\line(1,0){132.0000}}
\put(-1.0000,23.7143){\line(1,0){132.0000}}
\put(-1.0000,27.1429){\line(1,0){132.0000}}
\put(-1.0000,18.2857){\line(1,0){132.0000}}
\put(-1.0000,21.7143){\line(1,0){132.0000}}
\put(-1.0000,12.8571){\line(1,0){132.0000}}
\put(-1.0000,16.2857){\line(1,0){132.0000}}
\put(-1.0000,7.4286){\line(1,0){132.0000}}
\put(-1.0000,10.8571){\line(1,0){132.0000}}
\put(-1.0000,2.0000){\line(1,0){132.0000}}
\put(-1.0000,5.4286){\line(1,0){132.0000}}
\ligneepaisse
\color{blue}
\put(-1.0000,36.2857){\color{blue}\normalsize\makebox(0,0)[r]{rq\_in\_1}}
\put(0.0000,38.0000){\line(0,-1){3.4286}}
\put(4.3333,34.5714){\line(0,1){3.4286}}
\put(13.0000,38.0000){\line(0,-1){3.4286}}
\put(30.3333,34.5714){\line(0,1){3.4286}}
\put(39.0000,38.0000){\line(0,-1){3.4286}}
\put(65.0000,34.5714){\line(0,1){3.4286}}
\put(0.0000,34.5714){\line(1,0){4.3333}}
\put(4.3333,38.0000){\line(1,0){4.3333}}
\put(8.6667,38.0000){\line(1,0){4.3333}}
\put(13.0000,34.5714){\line(1,0){4.3333}}
\put(17.3333,34.5714){\line(1,0){4.3333}}
\put(21.6667,34.5714){\line(1,0){4.3333}}
\put(26.0000,34.5714){\line(1,0){4.3333}}
\put(30.3333,38.0000){\line(1,0){4.3333}}
\put(34.6667,38.0000){\line(1,0){4.3333}}
\put(39.0000,34.5714){\line(1,0){4.3333}}
\put(43.3333,34.5714){\line(1,0){4.3333}}
\put(47.6667,34.5714){\line(1,0){4.3333}}
\put(52.0000,34.5714){\line(1,0){4.3333}}
\put(56.3333,34.5714){\line(1,0){4.3333}}
\put(60.6667,34.5714){\line(1,0){4.3333}}
\put(65.0000,38.0000){\line(1,0){4.3333}}
\put(69.3333,38.0000){\line(1,0){4.3333}}
\put(73.6667,38.0000){\line(1,0){4.3333}}
\put(78.0000,38.0000){\line(1,0){4.3333}}
\put(82.3333,38.0000){\line(1,0){4.3333}}
\put(86.6667,38.0000){\line(1,0){4.3333}}
\put(91.0000,38.0000){\line(1,0){4.3333}}
\put(95.3333,38.0000){\line(1,0){4.3333}}
\put(99.6667,38.0000){\line(1,0){4.3333}}
\put(104.0000,38.0000){\line(1,0){4.3333}}
\put(108.3333,38.0000){\line(1,0){4.3333}}
\put(112.6667,38.0000){\line(1,0){4.3333}}
\put(117.0000,38.0000){\line(1,0){4.3333}}
\put(121.3333,38.0000){\line(1,0){4.3333}}
\put(125.6667,38.0000){\line(1,0){4.3333}}
\color{blue}
\put(-1.0000,30.8571){\color{blue}\normalsize\makebox(0,0)[r]{rq\_in\_2}}
\put(0.0000,32.5714){\line(0,-1){3.4286}}
\put(13.0000,29.1429){\line(0,1){3.4286}}
\put(21.6667,32.5714){\line(0,-1){3.4286}}
\put(39.0000,29.1429){\line(0,1){3.4286}}
\put(47.6667,32.5714){\line(0,-1){3.4286}}
\put(65.0000,29.1429){\line(0,1){3.4286}}
\put(0.0000,29.1429){\line(1,0){4.3333}}
\put(4.3333,29.1429){\line(1,0){4.3333}}
\put(8.6667,29.1429){\line(1,0){4.3333}}
\put(13.0000,32.5714){\line(1,0){4.3333}}
\put(17.3333,32.5714){\line(1,0){4.3333}}
\put(21.6667,29.1429){\line(1,0){4.3333}}
\put(26.0000,29.1429){\line(1,0){4.3333}}
\put(30.3333,29.1429){\line(1,0){4.3333}}
\put(34.6667,29.1429){\line(1,0){4.3333}}
\put(39.0000,32.5714){\line(1,0){4.3333}}
\put(43.3333,32.5714){\line(1,0){4.3333}}
\put(47.6667,29.1429){\line(1,0){4.3333}}
\put(52.0000,29.1429){\line(1,0){4.3333}}
\put(56.3333,29.1429){\line(1,0){4.3333}}
\put(60.6667,29.1429){\line(1,0){4.3333}}
\put(65.0000,32.5714){\line(1,0){4.3333}}
\put(69.3333,32.5714){\line(1,0){4.3333}}
\put(73.6667,32.5714){\line(1,0){4.3333}}
\put(78.0000,32.5714){\line(1,0){4.3333}}
\put(82.3333,32.5714){\line(1,0){4.3333}}
\put(86.6667,32.5714){\line(1,0){4.3333}}
\put(91.0000,32.5714){\line(1,0){4.3333}}
\put(95.3333,32.5714){\line(1,0){4.3333}}
\put(99.6667,32.5714){\line(1,0){4.3333}}
\put(104.0000,32.5714){\line(1,0){4.3333}}
\put(108.3333,32.5714){\line(1,0){4.3333}}
\put(112.6667,32.5714){\line(1,0){4.3333}}
\put(117.0000,32.5714){\line(1,0){4.3333}}
\put(121.3333,32.5714){\line(1,0){4.3333}}
\put(125.6667,32.5714){\line(1,0){4.3333}}
\color{blue}
\put(-1.0000,25.4286){\color{blue}\normalsize\makebox(0,0)[r]{rq\_in\_3}}
\put(21.6667,23.7143){\line(0,1){3.4286}}
\put(30.3333,27.1429){\line(0,-1){3.4286}}
\put(47.6667,23.7143){\line(0,1){3.4286}}
\put(56.3333,27.1429){\line(0,-1){3.4286}}
\put(65.0000,23.7143){\line(0,1){3.4286}}
\put(0.0000,23.7143){\line(1,0){4.3333}}
\put(4.3333,23.7143){\line(1,0){4.3333}}
\put(8.6667,23.7143){\line(1,0){4.3333}}
\put(13.0000,23.7143){\line(1,0){4.3333}}
\put(17.3333,23.7143){\line(1,0){4.3333}}
\put(21.6667,27.1429){\line(1,0){4.3333}}
\put(26.0000,27.1429){\line(1,0){4.3333}}
\put(30.3333,23.7143){\line(1,0){4.3333}}
\put(34.6667,23.7143){\line(1,0){4.3333}}
\put(39.0000,23.7143){\line(1,0){4.3333}}
\put(43.3333,23.7143){\line(1,0){4.3333}}
\put(47.6667,27.1429){\line(1,0){4.3333}}
\put(52.0000,27.1429){\line(1,0){4.3333}}
\put(56.3333,23.7143){\line(1,0){4.3333}}
\put(60.6667,23.7143){\line(1,0){4.3333}}
\put(65.0000,27.1429){\line(1,0){4.3333}}
\put(69.3333,27.1429){\line(1,0){4.3333}}
\put(73.6667,27.1429){\line(1,0){4.3333}}
\put(78.0000,27.1429){\line(1,0){4.3333}}
\put(82.3333,27.1429){\line(1,0){4.3333}}
\put(86.6667,27.1429){\line(1,0){4.3333}}
\put(91.0000,27.1429){\line(1,0){4.3333}}
\put(95.3333,27.1429){\line(1,0){4.3333}}
\put(99.6667,27.1429){\line(1,0){4.3333}}
\put(104.0000,27.1429){\line(1,0){4.3333}}
\put(108.3333,27.1429){\line(1,0){4.3333}}
\put(112.6667,27.1429){\line(1,0){4.3333}}
\put(117.0000,27.1429){\line(1,0){4.3333}}
\put(121.3333,27.1429){\line(1,0){4.3333}}
\put(125.6667,27.1429){\line(1,0){4.3333}}
\color{red}
\put(-1.0000,20.0000){\color{red}\normalsize\makebox(0,0)[r]{ack\_out\_1}}
\put(4.3333,18.2857){\line(0,1){3.4286}}
\put(13.0000,21.7143){\line(0,-1){3.4286}}
\put(30.3333,18.2857){\line(0,1){3.4286}}
\put(39.0000,21.7143){\line(0,-1){3.4286}}
\put(82.3333,18.2857){\line(0,1){3.4286}}
\put(86.6667,21.7143){\line(0,-1){3.4286}}
\put(95.3333,18.2857){\line(0,1){3.4286}}
\put(99.6667,21.7143){\line(0,-1){3.4286}}
\put(108.3333,18.2857){\line(0,1){3.4286}}
\put(112.6667,21.7143){\line(0,-1){3.4286}}
\put(121.3333,18.2857){\line(0,1){3.4286}}
\put(125.6667,21.7143){\line(0,-1){3.4286}}
\put(0.0000,18.2857){\line(1,0){4.3333}}
\put(4.3333,21.7143){\line(1,0){4.3333}}
\put(8.6667,21.7143){\line(1,0){4.3333}}
\put(13.0000,18.2857){\line(1,0){4.3333}}
\put(17.3333,18.2857){\line(1,0){4.3333}}
\put(21.6667,18.2857){\line(1,0){4.3333}}
\put(26.0000,18.2857){\line(1,0){4.3333}}
\put(30.3333,21.7143){\line(1,0){4.3333}}
\put(34.6667,21.7143){\line(1,0){4.3333}}
\put(39.0000,18.2857){\line(1,0){4.3333}}
\put(43.3333,18.2857){\line(1,0){4.3333}}
\put(47.6667,18.2857){\line(1,0){4.3333}}
\put(52.0000,18.2857){\line(1,0){4.3333}}
\put(56.3333,18.2857){\line(1,0){4.3333}}
\put(60.6667,18.2857){\line(1,0){4.3333}}
\put(65.0000,18.2857){\line(1,0){4.3333}}
\put(69.3333,18.2857){\line(1,0){4.3333}}
\put(73.6667,18.2857){\line(1,0){4.3333}}
\put(78.0000,18.2857){\line(1,0){4.3333}}
\put(82.3333,21.7143){\line(1,0){4.3333}}
\put(86.6667,18.2857){\line(1,0){4.3333}}
\put(91.0000,18.2857){\line(1,0){4.3333}}
\put(95.3333,21.7143){\line(1,0){4.3333}}
\put(99.6667,18.2857){\line(1,0){4.3333}}
\put(104.0000,18.2857){\line(1,0){4.3333}}
\put(108.3333,21.7143){\line(1,0){4.3333}}
\put(112.6667,18.2857){\line(1,0){4.3333}}
\put(117.0000,18.2857){\line(1,0){4.3333}}
\put(121.3333,21.7143){\line(1,0){4.3333}}
\put(125.6667,18.2857){\line(1,0){4.3333}}
\color{red}
\put(-1.0000,14.5714){\color{red}\normalsize\makebox(0,0)[r]{ack\_out\_2}}
\put(0.0000,16.2857){\line(0,-1){3.4286}}
\put(13.0000,12.8571){\line(0,1){3.4286}}
\put(21.6667,16.2857){\line(0,-1){3.4286}}
\put(39.0000,12.8571){\line(0,1){3.4286}}
\put(47.6667,16.2857){\line(0,-1){3.4286}}
\put(86.6667,12.8571){\line(0,1){3.4286}}
\put(91.0000,16.2857){\line(0,-1){3.4286}}
\put(99.6667,12.8571){\line(0,1){3.4286}}
\put(104.0000,16.2857){\line(0,-1){3.4286}}
\put(112.6667,12.8571){\line(0,1){3.4286}}
\put(117.0000,16.2857){\line(0,-1){3.4286}}
\put(125.6667,12.8571){\line(0,1){3.4286}}
\put(0.0000,12.8571){\line(1,0){4.3333}}
\put(4.3333,12.8571){\line(1,0){4.3333}}
\put(8.6667,12.8571){\line(1,0){4.3333}}
\put(13.0000,16.2857){\line(1,0){4.3333}}
\put(17.3333,16.2857){\line(1,0){4.3333}}
\put(21.6667,12.8571){\line(1,0){4.3333}}
\put(26.0000,12.8571){\line(1,0){4.3333}}
\put(30.3333,12.8571){\line(1,0){4.3333}}
\put(34.6667,12.8571){\line(1,0){4.3333}}
\put(39.0000,16.2857){\line(1,0){4.3333}}
\put(43.3333,16.2857){\line(1,0){4.3333}}
\put(47.6667,12.8571){\line(1,0){4.3333}}
\put(52.0000,12.8571){\line(1,0){4.3333}}
\put(56.3333,12.8571){\line(1,0){4.3333}}
\put(60.6667,12.8571){\line(1,0){4.3333}}
\put(65.0000,12.8571){\line(1,0){4.3333}}
\put(69.3333,12.8571){\line(1,0){4.3333}}
\put(73.6667,12.8571){\line(1,0){4.3333}}
\put(78.0000,12.8571){\line(1,0){4.3333}}
\put(82.3333,12.8571){\line(1,0){4.3333}}
\put(86.6667,16.2857){\line(1,0){4.3333}}
\put(91.0000,12.8571){\line(1,0){4.3333}}
\put(95.3333,12.8571){\line(1,0){4.3333}}
\put(99.6667,16.2857){\line(1,0){4.3333}}
\put(104.0000,12.8571){\line(1,0){4.3333}}
\put(108.3333,12.8571){\line(1,0){4.3333}}
\put(112.6667,16.2857){\line(1,0){4.3333}}
\put(117.0000,12.8571){\line(1,0){4.3333}}
\put(121.3333,12.8571){\line(1,0){4.3333}}
\put(125.6667,16.2857){\line(1,0){4.3333}}
\color{red}
\put(-1.0000,9.1429){\color{red}\normalsize\makebox(0,0)[r]{ack\_out\_3}}
\put(21.6667,7.4286){\line(0,1){3.4286}}
\put(30.3333,10.8571){\line(0,-1){3.4286}}
\put(47.6667,7.4286){\line(0,1){3.4286}}
\put(56.3333,10.8571){\line(0,-1){3.4286}}
\put(65.0000,7.4286){\line(0,1){3.4286}}
\put(82.3333,10.8571){\line(0,-1){3.4286}}
\put(91.0000,7.4286){\line(0,1){3.4286}}
\put(95.3333,10.8571){\line(0,-1){3.4286}}
\put(104.0000,7.4286){\line(0,1){3.4286}}
\put(108.3333,10.8571){\line(0,-1){3.4286}}
\put(117.0000,7.4286){\line(0,1){3.4286}}
\put(121.3333,10.8571){\line(0,-1){3.4286}}
\put(0.0000,7.4286){\line(1,0){4.3333}}
\put(4.3333,7.4286){\line(1,0){4.3333}}
\put(8.6667,7.4286){\line(1,0){4.3333}}
\put(13.0000,7.4286){\line(1,0){4.3333}}
\put(17.3333,7.4286){\line(1,0){4.3333}}
\put(21.6667,10.8571){\line(1,0){4.3333}}
\put(26.0000,10.8571){\line(1,0){4.3333}}
\put(30.3333,7.4286){\line(1,0){4.3333}}
\put(34.6667,7.4286){\line(1,0){4.3333}}
\put(39.0000,7.4286){\line(1,0){4.3333}}
\put(43.3333,7.4286){\line(1,0){4.3333}}
\put(47.6667,10.8571){\line(1,0){4.3333}}
\put(52.0000,10.8571){\line(1,0){4.3333}}
\put(56.3333,7.4286){\line(1,0){4.3333}}
\put(60.6667,7.4286){\line(1,0){4.3333}}
\put(65.0000,10.8571){\line(1,0){4.3333}}
\put(69.3333,10.8571){\line(1,0){4.3333}}
\put(73.6667,10.8571){\line(1,0){4.3333}}
\put(78.0000,10.8571){\line(1,0){4.3333}}
\put(82.3333,7.4286){\line(1,0){4.3333}}
\put(86.6667,7.4286){\line(1,0){4.3333}}
\put(91.0000,10.8571){\line(1,0){4.3333}}
\put(95.3333,7.4286){\line(1,0){4.3333}}
\put(99.6667,7.4286){\line(1,0){4.3333}}
\put(104.0000,10.8571){\line(1,0){4.3333}}
\put(108.3333,7.4286){\line(1,0){4.3333}}
\put(112.6667,7.4286){\line(1,0){4.3333}}
\put(117.0000,10.8571){\line(1,0){4.3333}}
\put(121.3333,7.4286){\line(1,0){4.3333}}
\put(125.6667,7.4286){\line(1,0){4.3333}}
\color{red}
\put(-1.0000,3.7143){\color{red}\normalsize\makebox(0,0)[r]{gr\_out}}
\put(0.0000,2.0000){\line(0,1){3.4286}}
\put(4.3333,5.4286){\line(0,-1){3.4286}}
\put(56.3333,2.0000){\line(0,1){3.4286}}
\put(65.0000,5.4286){\line(0,-1){3.4286}}
\put(0.0000,5.4286){\line(1,0){4.3333}}
\put(4.3333,2.0000){\line(1,0){4.3333}}
\put(8.6667,2.0000){\line(1,0){4.3333}}
\put(13.0000,2.0000){\line(1,0){4.3333}}
\put(17.3333,2.0000){\line(1,0){4.3333}}
\put(21.6667,2.0000){\line(1,0){4.3333}}
\put(26.0000,2.0000){\line(1,0){4.3333}}
\put(30.3333,2.0000){\line(1,0){4.3333}}
\put(34.6667,2.0000){\line(1,0){4.3333}}
\put(39.0000,2.0000){\line(1,0){4.3333}}
\put(43.3333,2.0000){\line(1,0){4.3333}}
\put(47.6667,2.0000){\line(1,0){4.3333}}
\put(52.0000,2.0000){\line(1,0){4.3333}}
\put(56.3333,5.4286){\line(1,0){4.3333}}
\put(60.6667,5.4286){\line(1,0){4.3333}}
\put(65.0000,2.0000){\line(1,0){4.3333}}
\put(69.3333,2.0000){\line(1,0){4.3333}}
\put(73.6667,2.0000){\line(1,0){4.3333}}
\put(78.0000,2.0000){\line(1,0){4.3333}}
\put(82.3333,2.0000){\line(1,0){4.3333}}
\put(86.6667,2.0000){\line(1,0){4.3333}}
\put(91.0000,2.0000){\line(1,0){4.3333}}
\put(95.3333,2.0000){\line(1,0){4.3333}}
\put(99.6667,2.0000){\line(1,0){4.3333}}
\put(104.0000,2.0000){\line(1,0){4.3333}}
\put(108.3333,2.0000){\line(1,0){4.3333}}
\put(112.6667,2.0000){\line(1,0){4.3333}}
\put(117.0000,2.0000){\line(1,0){4.3333}}
\put(121.3333,2.0000){\line(1,0){4.3333}}
\put(125.6667,2.0000){\line(1,0){4.3333}}
\end{picture}
}


\subsection{Vérifications}

Nous cherchons maintenant à vérifier un certain nombre de propriétés sur l'arbitre.\\

Afin de vérifier la propriété suivante : P1 : A tout instant, il n'y a qu'une seule
requête autorisée à accéder au bus, nous implémentons l'observeur ci-dessous.

\begin{verbatim}
node observer_P1(ack_out_1, ack_out_2, ack_out_3 : bool) returns (res : bool);
let
  res = if ack_out_1 then not (ack_out_2 or ack_out_3) else
          if ack_out_2 then not (ack_out_1 or ack_out_3) else
            if ack_out_3 then not (ack_out_1 or ack_out_2) else
              true;
tel

node verifier_P1(rq_in_1, rq_in_2, rq_in_3 : bool) returns (res : bool);
var ack_out_1, ack_out_2, ack_out_3, gr_out : bool;
let
  ack_out_1, ack_out_2, ack_out_3, gr_out = arbitre_3(rq_in_1, rq_in_2, rq_in_3);

  res = observer_P1(ack_out_1, ack_out_2, ack_out_3);
tel
\end{verbatim}

L'outils Lesar détermine cette propriété P1 comme VRAIE.\\

Pour vérifier la propriété P2 : A tout instant, une autorisation d'accès au bus correspond bien à une demande effective du bus, nous définissons l'observeur ci-dessous.
\begin{verbatim}
node observer_P2(rq_in_1, rq_in_2, rq_in_3, ack_out_1, ack_out_2, ack_out_3 : bool)
returns (res : bool);
let
    res = if ack_out_1 then rq_in_1 else
            if ack_out_2 then rq_in_2 else
              if ack_out_3 then rq_in_3 else
                true;
tel

node verifier_P2(rq_in_1, rq_in_2, rq_in_3 : bool) returns (res : bool);
var ack_out_1, ack_out_2, ack_out_3, gr_out : bool;
let
  ack_out_1, ack_out_2, ack_out_3, gr_out = arbitre_3(rq_in_1, rq_in_2, rq_in_3);

  res = observer_P2(rq_in_1, rq_in_2, rq_in_3, ack_out_1, ack_out_2, ack_out_3);
tel
\end{verbatim}

L'outils Lesar détermine cette propriété P2 comme VRAIE.

\section{Indeterminisme et quasi-synchronisme}

\subsection{Indeterminisme}

Dans l'énoncé le code correspondant à la valeur \texttt{reset} est
manquant. Le code complet est le suivant :

\begin{verbatim}
node env(S,R : bool; o : bool) returns (set,reset : bool);
let
  set = S and not R or (S and R and o);
  reset = R and not S or (S and R and not o);
tel
\end{verbatim}

Les valeurs de \texttt{set} et \texttt{reset} correspondent
respectivement aux formules $S \wedge \neg R$ et $\neg S \wedge R$ qui
ne peuvent pas être vrais en même temps. Par contre elles peuvent être
fausses en même temps ($R = S$). Si $R = S = vrai$, alors $set = o = \neg
\neg o = \neg reset$. On respecte donc la première
propriété. Sinon \texttt{set} et \texttt{reset} sont faux
(ce qui nous donne la première configuration)

On ne peut par contre pas vérifier le fait qu'à chaque instant le n\oe
ud Lustre puisse générer les trois configurations grâce à un
observateur. Si on utilise un observateur pour cela, il se pourraît que
les données en entrée soient telles que l'une des trois configurations
ne se produise jamais. Et donc nous n'aurions jamais la réponse, un
peu à la manière d'un problème indécidable. Nous sommes face à un problème
d'atteignabilité, alors que nous n'abordions jusque là que la propriété de
sûreté. On ne peut pas vérifier qu'un état est atteignable
dans le cas présent et avec les outils à notre disposition, parce
qu'un observateur va vérifier une propriété globale sur les états. Il
en est de même pour le model checker \texttt{xlesar} qui s'arrête si
une propriété est fausse. Il faudrait en fait ajouter la possibilité
d'exprimer des points fixes dans les formules.

\subsection{Quasi-synchronisme}


\end{document}

# Local Variables:
# compile-command: "rubber -d rapport.tex"
# End:
